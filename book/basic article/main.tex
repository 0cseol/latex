\documentclass{article} % 문서 클래스: 기본 article 사용
% \documentclass[oneside]{book} - 기본 양면구성이라 단면으로 하려면 oneside 입력
\usepackage[ % 만약 일괄적용하고 싶으면 margin=2.5cm 이런식으로 쓰면 됨
  a4paper, % a4 용지, b4 등으로 변경 가능
  top=2.5cm,
  bottom=2.5cm,
  left=2.5cm,
  right=1.5cm
]{geometry}

\usepackage{comment} % 여러 줄 주석을 위해 포함함 보통은 필요 없음
\begin{comment}
위의 article에 대해서 설명하면, 다음과 같이 세부 설정 가능함
\documentclass[A4paper, 11pt, fleqn]{article}
A4paper로 지정(프로그램에서 설정되어있긴 함), 11pt(안쓰면 기본 10pt) flush left equations(기본 수식 왼쪽정렬)

** article – 논문, 보고서 등에 쓰이며 짧은 문서에 적합함
report – 장(chapter)이 있는 중·장기 보고서나 과제에 적합함
** book – 책이나 논문집처럼 장, 절, 부 구성이 필요한 긴 문서에 사용됨
** beamer – 발표 슬라이드 제작용으로 애니메이션과 전환 효과 지원함
slides – 구식 슬라이드 형식 문서에 사용되며 현재는 거의 사용되지 않음
letter – 편지 형식 문서 작성에 사용됨
proc – 학회 논문집 형식에 적합하도록 article을 기반으로 확장됨
memoir – book/report/article을 통합한 고급 사용자용 클래스
IEEEtran – IEEE 학술지 형식에 맞춘 클래스 (학술 논문 제출용)
\end{comment}

\usepackage{kotex}      % 한글 사용시 필수
\usepackage{amsmath}    % 수식 작성(정렬 등 포함)
\usepackage{amssymb}    % 수학 기호 확장
\usepackage{graphicx}   % 이미지 삽입 가능
\usepackage{hyperref}   % 하이퍼링크(목차, URL 등)
\usepackage{fancyhdr}   % 머리말, 꼬리말
\usepackage{xcolor}     % 글자색 변경(color 보다 색 혼합 등 기능 많은 xcolor 더 많이 씀)
\DeclareGraphicsExtensions{.pdf, .png, .jpg}
% 그림파일 확장자 생략할 수 있도록 함(pdf, png, jpg 순으로 검색)

\hbadness=10000 % underfull hbox 경고 무시(수평 방향 여백 경고 badness 10000까지 무시)
\vbadness=10000 % overfull hbox 경고 무시(수직 방향 페이니 나눔 단락 경고 badness 10000까지 무시)
\hfuzz=5pt % 수평박스 커서 오른쪽 여백 넘어가도 5pt까지 무시
%디버깅 스트레스 줄일 때 유용하나, 완성본에서는 다시 살려서 조정하는 것이 좋을 수도 있음

\pagestyle{fancy} % 머리말 활성화 - fancyhdr 패키지 사용해야만 함
\lhead{상단 왼쪽}
\chead{상단 가운데}
\rhead{상단 오른쪽}
\lfoot{하단 왼쪽}
\cfoot{하단 가운데}
\rfoot{\today\\오늘날짜입니다} % 하단 오른쪽

%타이틀 만들기(첫페이지에 해당함)
\title{article sample 입니다.} % 문서 제목
\author{홍길동} % 저자
\date{\today} % 날짜를 생략하고 싶으면 \today 만 지워야 함. 기본 설정이 날짜가 나오기 때문임.


\begin{document}
\maketitle % 타이틀이 있는 경우 머리말은 첫페이지 생략됨
% 첫페이지까지 머리말 나오게 하려면 \thispagestyle{fancy} 입력

이 글은 \LaTeX{}을 이용하여 작성한 것입니다. \\
% 줄바꿈 이후 빈줄을 넣으면 아래에 경고 메시지가 뜸. 무시해도 되지만 LaTeX이 문단구분을 못하므로 해결하는 것이 좋음

\textbf{\textcolor{blue}{오일러 공식}}을 이용한 2배각, 3배각 \dots\\ % 굵은 글씨 bf, 파란색color, 생략부호dot(...)
\rule{\linewidth}{1pt}\\ % 사각형 만들기는 \rule{가로길이}{가로굵기}, linewidth은 현재 줄 너비(문단 전체 너비)
% 즉, 문단 너비만큼 길고 두께가 1인 수평선이 만들어짐. 1pt, 2cm등 입력 가능. 정사각형도 만들 수 있음.

타이틀을 만들고, 타이틀 페이지에서 하고 싶은 말을 쓰고 있습니다.\\
빈줄을 하나 더 넣으면 단락이 바뀌어서 들여쓰기가 됩니다. 들여쓰기는 우측의 기호를 사용하고 있네요. \\
\\
`\verb|\\newline|'(또는 ``\verb|\\|'')을 중간에 삽입하면 들여쓰기하지 않고 새로운 줄이 삽입되게 할 수 있습니다. \\
(verb를 이용하여 함수가 아닌 문자 자체를 그대로 출력하도록 했네요. 여는따옴표와 닫는 따옴표도 주목하세요.) \\

여기에서는 제목페이지에서 하고 싶은 말을 작성합니다.

줄바꿈 문단부호가 없어도 그냥 엔터를 두번 치면 이렇게 문단이 분리됩니다.
하지만 이렇게 그냥 한줄만 엔터치면
연달아 글이 써지겠지요? \\\\
이렇게 네번을 치면 문단이 분리되지만 들여쓰기가 되지 않습니다. 그저 줄만 바꾸는 것입니다.\\
\vspace{2\baselineskip} % 2줄*1.2배 띄우는 효과(정확히 두줄 띄우려면 2em으로 씀) - 바깥 문단에서 써야 함.
\vspace{5pt} % 포인트 기준. 1cm 등도 가능
기본 설정에서 첫 페이지에서는 설정된 타이틀이 나오고, 머리말이 나오지 않습니다.\\

\begin{center} % 가운데 정렬 - 안쓸 것임
가운데 정렬은 `\verb|\begin{center}|'을 쓸 수 있지만\\
문단 환경을 써서 불필요한 줄 늘림이 생깁니다.
\end{center}
{\centering 이렇게 \Large \textbf{가운데 정렬} \par \normalsize을 하면 \\ % 글씨 크게 하고 리셋
더 간단하고 깔끔하게 할 수 있습니다.\par}

\vspace{2em}
다음은 그림을 삽입해보겠습니다. 사전에 graphicx 패키지를 호출해야 합니다.\\

\begin{figure}[h] % 여기(here)에 그림 배치 - 넘버링됨. 넘버링 안되게 하려면 바로 이미지 삽임
% 아무 옵션 없으면 적절한 위치에 자동 배치되어버림 - 위(t), 아래(b), 그림전용 페이지(p), 다음페이지로 넘어가기도 함
\centering % 가운데 정렬
\includegraphics[scale=0.5]{cat} % 위에서 그림파일 확장자 생략하였음
% cat이미지(확장자는 pdf, png 순으로 탐색하도록 위에서 설정)를 50% 크기로 삽입
% 사진이 없어서 오류가 나지만 사각 박스로 만들어지긴 함
\caption{catttt고양} % 그림 아래 설명(캡션)
\label{fig:cat} %추후 \ref 로 참조할 때 사용할 라벨. 캡션 뒤에 위치해야 함
\end{figure}

라벨을 그림(fig), 표(tab), 식(eq) 등으로 명명하고 관리하면 추후 매우 편리합니다.\\
예를 들어, 그림 \ref{fig:cat}을 불러와서 이에 대한 설명을 할 수 있습니다. \\
또는 autoref를 이용하면 \autoref{fig:cat}을 이용할 수도 있습니다. \\

\begin{flushright} % 오른쪽 정렬
다음 페이지로 넘어갑시다.
\end{flushright}

\hfill 사실 한 줄만 오른쪽 정렬은 이게 더 편합니다.

\newpage % 다음 페이지

계층 구조에 대하여 알아봅시다.
\part{수준 1을 작성합니다.}
	% \chapter{수준 2를 작성합니다.} - book이나 report에서만 사용함. article에선 사용하지 않음.
  	\section{수준 3을 작성합니다.}
    	\subsection{수준 4를 작성합니다.}
      	\subsubsection{수준 5를 작성합니다.}
        	\paragraph{수준 6을 작성합니다.}
          수준 6과 7은 바로 이야기를 시작할 수 있습니다.\\번호가 붙지 않습니다.
          	\subparagraph{수준 7을 작성합니다.} 마찬가지로 이야기를 작성할 수 있습니다. \\\\

이상 계층 구조에 대하여 알아보았습니다.
\vspace{2em}

\section{수식 알아보기}

이제는 수식에 대하여 알아볼 것입니다.\\

글자 중간에 수식을 넣고 싶은 경우 `\verb|$$|' 사이에 수식을 입력할 수 있습니다.\\
분수를 예로 들면, 분수는 `\verb|\frac{분자}{분모}|'을 이용할 수 있는데,\\
자동 크기 조절 분수는: $\frac{1}{2}$, 
디스플레이용(작은 환경에서도 커짐) 분수는: $\dfrac{1}{2}$, 
텍스트용 분수는: $\tfrac{1}{2}$로 나타냅니다.\\
에너지 식을 $E=\dfrac{{\hbar^{2} k^{2}}}{2m}$로 나타낼 수 있습니다.\\

\begin{equation} % 한줄 수식. 넘버링이 자동으로 생성됨
    E = mc^2
\end{equation}

\begin{equation*} % 위와 동일하지만 넘버링 되지 않음
    E = mc^2
\end{equation*}

\begin{equation}
    \sin \theta = \frac{e^{i \theta} - e^{-i \theta}}{2i} % 자동 크기 분수
\end{equation}

\[ % 넘버링 포함되지 않은 수식은 이렇게 나타낼 수도 있음. 하지만 *이 있으면 추후 관리가 더 편함
    e^{2i \theta} = {e^{i \theta}}^2
\]

$$ a=\frac{F}{m} $$ % 넘버링 포함되지 않은 수식은 마찬가지로 이렇게 나타낼 수도 있음. 엔터쳐도 됨

\begin{align}
a &= b + c\\
x &= y + z\\
&=syc \notag\\ % 넘버링 제거
&=nch
\end{align}
% 여러 줄 수식 각줄마다 넘버링 됨.
% 전체 넘버링 포함하지 않으려면 * 붙임.
% 하나만 포함되지 않게 하려면 \notag \\ 붙임.
% &으로 기준을 잡으면 자동 정렬되고, 정렬 기준은 하나만 잡는 것이 좋음.
% 여러 & 사용하려면 alignat 사용

\vspace{2em}

\begin{center}
응용 수식이 있는 다음 (페이지)로 갑시다.
\end{center}

\newpage

대괄호, 소괄호가 글자 크기에 맞게 작성되게 하려면 `\verb|\left(|'나 `\verb|\left[|'를 이용합니다.\\
\begin{flalign} % fl(flushleft)을 붙여서 왼쪽 정렬. 수식 맨 뒤에 & 붙여줘야 정렬됨.(맨 앞으로 정렬하는 셈)
    \cos^3 \theta &= \left[ \frac{1}{2} \left( e^{i \theta} + e^{-i \theta} \right) \right]^{3} \\
    &= \frac{1}{4} \left[ \frac{1}{2} \left( e^{3i \theta} + e^{-3i \theta} \right) + \frac{3}{2} \left(e^{i \theta} + e^{-i \theta} \right) \right]\\
    &= \frac{1}{4} \left[ \cos 3 \theta + 3 \cos \theta \right] &
\end{flalign}

\begin{align}
    I&= \int^{2\pi}_{0} \int^{R_{2}}_{R_{1}}\sigma r^{3}drd \theta \\
    &= \sigma \cdot \frac{1}{4} \left( R^{4}_{2}-R^{4}_{1} \right) \cdot 2x \notag\\
    &= \frac{1}{2} \cdot \sigma \pi \left( R^{2}_{2}-R^{2}_{1} \right) \cdot \left( R^{2}_{2}+R^{2}_{1} \right) \notag\\
    &= \frac{1}{2} M \left( R^{2}_{2}+R^{2}_{1} \right)
\end{align}

\begin{equation}
    \therefore I =\frac{1}{2} M \left( R_{1}^{2}+R_{2}^{2} \right)
    \label{eq:moment_of_inertia_thinring} % 라벨링
\end{equation}

참조는 \ref{eq:moment_of_inertia_thinring}로 할 수 있고, \\
괄호가 나오게 하려면, `식 \eqref{eq:moment_of_inertia_thinring}'과 같이 참조할 수 있습니다. \\
또는 autoref를 이용하여 \autoref{eq:moment_of_inertia_thinring}과 같이 참조할 수 있습니다. \\

오늘은 \autoref{eq:moment_of_inertia_thinring}\을 살펴보았다.

\begin{align}
    I=\int r^2 dm \\
    \iint \\
    \oint_{A} \\
    \int^{2}_{2}
\end{align}





\vspace{2em}

\begin{center}
실제 책 작성의 예시가 있는 다음 페이지로 갑시다.
\end{center}

\newpage

\section{Introduction}
$\textrm{For Ampere's law,}$

\begin{equation}
    A_1 = \frac{\mu_0 i_1}{4 \pi} \oint { \frac{d\vec{s_1}}{r} }
\end{equation}

이고, inductor 1이 2에 가하는 자속 $ \Phi_{21}$은 Stokes' theorem에 따라,

\begin{equation}
    \Phi_{21} = \int \triangledown \times \vec{A_1} \cdot d\vec{A_2} = \oint \vec{A_1} \cdot d\vec{s_2}
\end{equation}

가 된다. 또한 Mutual Inductance $M_{21}$를 inductance와 마찬가지로

\begin{equation*}
    M_{21} = \frac{\Phi_{21}}{i_1}
\end{equation*}
이라고 정의하면,

\begin{equation}
    M_{21} = \frac{\mu_0}{4\pi} \oint \oint \frac{d\vec{s_1} \cdot d\vec{s_2}}{r}
\end{equation}

이 되므로 순서를 바꿔도 값이 똑같다는 것을 알 수 있다.
따라서 $ M_{21} = M_{12} $가 된다. \\

\begin{equation}
    5 \times 10^{-6}\,\mathrm{J}
\end{equation}

\end{document}



\end{document}
