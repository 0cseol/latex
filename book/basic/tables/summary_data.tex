%% 표 1: 유효숫자 판별 규칙 요약
\newcommand{\summarytableA}{
\begin{table}[htbp]
\centering
\begin{tabular}{|c|l|}
\hline
\textbf{구분} & \textbf{설명} \\
\hline
규칙 1 & 0이 아닌 숫자는 모두 유효숫자 \\
규칙 2 & 유효숫자 사이의 0은 유효숫자 \\
규칙 3 & 소수점이 있는 경우 끝의 0도 유효숫자 \\
규칙 4 & 소수점이 없는 경우 끝의 0은 유효숫자 아님 \\
\hline
\end{tabular}
\caption{유효숫자 판별 규칙 요약}
\end{table}
}

\newcommand{\summarytableB}{
\begin{table}[htbp]
\centering
\begin{tabular}{|c|l|}
\hline
\textbf{연산 종류} & \textbf{유효숫자 유지 기준} \\
\hline
덧셈/뺄셈 & 소수점 기준, 가장 적은 소수점 자리수 \\
곱셈/나눗셈 & 전체 유효숫자 수 기준, 가장 적은 유효숫자 \\
\hline
\end{tabular}
\caption{계산 시 유효숫자 처리 기준}
\end{table}
}