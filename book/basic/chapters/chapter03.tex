\chapter{연습}
\section{첫 번째 연습}
\blue{블루 blue} \red{레드 red} \gray{그레이 gray} \\
\blueb{블루볼드 blueb} \redb{레드볼드 redb} \grayb{그레이볼드 grayb} \\
\bd{볼드 bd} \itl{이탤릭 itl} \\
\memo{나만보는 memo} \\
\seolf{seolf함수} \\
\hl{형광펜 하이라이트 hl} \hlbold{형관펜 볼드 hlbold} 여러 줄 여러 줄 여러 줄 여러 줄 \\
\centertext{여러 줄\\가운데정렬}
바로 다음 이어서 쓰기, 바로 다음 이어서 쓰기, 바로 다음 이어서 쓰기, 바로 다음 이어서 쓰기

\anum\ anum은 이렇게 쓰고 \\
\anum\ 이렇게 쓰고 \\
중간에 내용 적었다가 \\
\anum\ 다시 쓰고 \\
\setcounter{anumcounter}{0}
초기화 했다가 \\
\anum\ 다시 쓰고 \\
\textbf{\anum\ 굵게 지정해서 다시 쓰고} \\
중간에 내용 적었다가 \\
\anum\ 다시 쓰고 \\
\bnum\ bnum은 이렇게 쓰고 \\
\bnum\ 다시 쓰고 \\
\setcounter{bnumcounter}{0}
초기화 했다가 \\
\bnum\ 다시 쓰고 \\

\newpage

\section{두 번째 연습은 3.1 처럼 들어감 - 페이지마다 뜨기도 함}
\subsection{이건 3.1.1 처럼 들어감 여기까지 목차 정리됨}
\subsubsection{이건 3.1.1.1 처럼 들어가야하는데 번호 안뜨고, 목차에 안들어감}

다음페이지 \\
\newpage

글자만 \grayinline{박스로} 채우기
\begin{graybox}
회색 박스
\end{graybox}

\begin{redbox}
빨간배경 박스 \\
redbox
\end{redbox}

\begin{bluebox}
파란 배경 박스 \\
bluebox
\end{bluebox}

\newpage
\subsection{표 정리}

간단한 표는 내부에 삽입하지만, 중요하고 참조할만한 것은 table에 모아둡니다. \\
LCR로 지정한 표의 모습 \\
\vspace{-1em} \\ % 이정도 여백 있어야 적당히 보이는 것 같음
\begin{tabular}{|L{3cm}|C{3cm}|R{5cm}|} % 열 3개: 폭 3, 3, 5cm 정렬은 LCR, 테두리 있음
\hline
\graycell{구분} & \bluecell{내용} & \bluecell{내용} \\
\hline
\redcell{경고} & 중요 사항 & \\
\hline
\end{tabular}

tab 함수를 새로 만들어서 지정하였음 \\
\tab{|L{3cm}|C{3cm}|R{5cm}|}{
\graycell{구분} & \bluecell{내용} & \bluecell{내용} \\
\hline
\raggedleft 오른쪽 정렬 & \raggedright 다시 왼쪽 정렬 \tabularnewline
}

p형태로 셀마다 정렬 따로한 모습 \\
\vspace{-2.1em} % 가운데 정렬시 이정도 여백
\begin{center}
\begin{tabular}{|p{4cm}|p{4cm}|}
\hline
\raggedright 왼쪽 정렬 & \centering 가운데 정렬 \tabularnewline
\hline
\raggedleft 오른쪽 정렬 & \raggedright 다시 왼쪽 정렬 \tabularnewline
\hline
\end{tabular}
\end{center}

ctab 함수를 새로 만들어서 지정하였음 \\
\ctab{|p{4cm}|p{4cm}|}{
\raggedright 왼쪽 정렬 & \centering 가운데 정렬 \tabularnewline
\hline
\raggedleft 오른쪽 정렬 & \raggedright 다시 왼쪽 정렬 \tabularnewline
}


표에 label을 달기 위해서는 table 안에 tabular를 넣고, caption, label을 씁니다.
\begin{table}[htbp] % 표 float 환경 시작
  \centering % 가운데 정렬
  \begin{tabular}{|c|c|}
    \hline
    항목 & 내용 \\
    \hline
    A & 설명 \\
    \hline
  \end{tabular}
  \caption{설명용 표 제목. 번호는 chapter에 맞게 부여됨}
  \label{tab:example} % 추후 참조 가능
\end{table}

\biglabel{12.} 빅라벨 사용 방법 big label

\newpage
\subsection{표의 결론}

tab 함수를 새로 만들어서 지정하였음 \\
\tab{|L{3cm}|C{3cm}|R{5cm}|}{
\graycell{구분} & \redcell{내용} & \bluecell{내용} \\
\hline
\leftcell{왼쪽} & \centercell{가운데} & \rightcell{오른쪽} \\
\hline
\bluecell{\leftcell{왼쪽}} & graycell{\centercell{가운데}} & \redcell{\rightcell{오른쪽}} \\
}
\\\\
ctab(가운데정렬) 함수를 새로 만들어서 지정하였음 \\
\ctab{|p{4cm}|p{4cm}|}{
\leftcell{왼쪽} & \centercell{가운데}  \\
\hline
\centercell{가운데} & \rightcell{오른쪽} \\
}
table 만든 것 \\
\begin{table}[htbp] % 표 float 환경 시작
  \ctab{|p{4cm}|p{4cm}|}{
  \leftcell{왼쪽} & \centercell{가운데}  \\
  \hline
  \centercell{가운데} & \rightcell{오른쪽} \\}
  \caption{설명용 표 제목. 번호는 chapter에 맞게 부여됨}
  \label{tab:example1} % 추후 참조 가능
\end{table}

참조는 `표 \ref{tab:example1}' 이런식으로 할 수 있음