\chapter{서론}

\section{연구의 배경과 목적}
현대 과학 기술의 발달은 교육 현장에 다양한 변화를 일으키고 있으며, 그중에서도 물리 교육은 학생들의 과학적 사고력과 문제 해결력을 기르는 데 중요한 역할을 함. 본 장에서는 연구의 동기와 목적을 간략히 설명함.

\section{선행 연구}
관련된 선행 연구들을 검토함으로써 본 연구의 차별성과 의의를 밝히고자 함. 특히 \seolf{유효숫자} 개념의 실제 활용에 주목한 이전 연구들을 비교함.

\section{연구 내용 개요}
본 책에서는 총 다섯 장에 걸쳐 과학영재 학생들의 유효숫자 활용 현황을 분석하고, 그 교육적 배경을 고찰함. 이후 장에서는 구체적인 분석 결과와 함께 시사점을 도출함.
해당 개념은 \cite{halliday2022}, \cite{kim2021sigfig}에서 자세히 다루고 있다.


\section{유효숫자 처리 요약}

유효숫자는 수치의 신뢰도를 나타내는 지표로, 측정값의 자릿수에 대한 정보를 제공한다. 먼저 유효숫자가 무엇인지, 어떻게 판단하는지를 요약하면 다음과 같다.
\newpage
% 표 정의 불러오기
%% 표 1: 유효숫자 판별 규칙 요약
\newcommand{\summarytableA}{
\begin{table}[htbp]
\centering
\begin{tabular}{|c|l|}
\hline
\textbf{구분} & \textbf{설명} \\
\hline
규칙 1 & 0이 아닌 숫자는 모두 유효숫자 \\
규칙 2 & 유효숫자 사이의 0은 유효숫자 \\
규칙 3 & 소수점이 있는 경우 끝의 0도 유효숫자 \\
규칙 4 & 소수점이 없는 경우 끝의 0은 유효숫자 아님 \\
\hline
\end{tabular}
\caption{유효숫자 판별 규칙 요약}
\end{table}
}

\newcommand{\summarytableB}{
\begin{table}[htbp]
\centering
\begin{tabular}{|c|l|}
\hline
\textbf{연산 종류} & \textbf{유효숫자 유지 기준} \\
\hline
덧셈/뺄셈 & 소수점 기준, 가장 적은 소수점 자리수 \\
곱셈/나눗셈 & 전체 유효숫자 수 기준, 가장 적은 유효숫자 \\
\hline
\end{tabular}
\caption{계산 시 유효숫자 처리 기준}
\end{table}
}

% 표 1 삽입
\summarytableA

위의 규칙을 바탕으로 유효숫자를 판단할 수 있다. 그러나 실제 계산에서는 연산 종류에 따라 처리 기준이 달라진다.

% 표 2 삽입
\summarytableB

이처럼 유효숫자 처리는 단순히 숫자를 읽는 규칙이 아닌, 상황별 판단이 필요하다.
