\chapter{챕터 내용 작성}

\section{유효숫자란 무엇인가}

물리학에서 수치를 다룰 때 유효숫자는 단순한 자릿수가 아닌, 측정의 신뢰도를 의미함. 예를 들어, $1.23$은 세 자리 유효숫자를 가지며 이는 각 자리의 의미 있는 측정값을 포함함을 뜻함.  

다음은 유효숫자의 예시를 \seolf{정리}해보자.

\begin{itemize}
  \item \seolf{3.40} $\Rightarrow$ 유효숫자 \seolf{3}자리
  \item \seolf{0.00450} $\Rightarrow$ 유효숫자 \seolf{3}자리
  \item \seolf{200} $\Rightarrow$ 유효숫자 \seolf{1}~\seolf{3}자리 (표기 방식에 따라 다름)
\end{itemize}

\section{학습 포인트 정리}


\section{학습 활동 예시}

다음 문장에서 빈칸을 채워보자:

"실험값 $4.00$은 유효숫자 \seolf{3}자리를 가진다."

혹은 다음과 같이 제시할 수 있다:

"실험값 $4.00$은 유효숫자 \seolf{   }자리를 가진다."


\clearpage
