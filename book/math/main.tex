
\documentclass[10pt]{book}
% 공통
\usepackage{xparse} % LaTeX3 언어 핵심 함수 제공 - 함수형 스타일, 내부 처리
\usepackage{expl3} % exp13 위에 구축된 고급 명령어 정의
\usepackage[hidelinks]{hyperref} % 참조나 목차에서 하이퍼링크가 가능하게 함
\usepackage{xifthen} % if 조건문 사용

% 폰트
\usepackage{luatexko} % 한글 폰트 사용. kotex의 lua 전용 후속 버전
\usepackage{fontspec} % 글꼴 강제 기울이기(한글 폰트 기울임)

% 색상
\usepackage{xcolor} % 색상
\usepackage{transparent} % lua만 됨. xcolor필요
\usepackage{luacolor,lua-ul} % 한글 형광펜 줄바꿈 등 완전 지원

% 이미지
\usepackage{graphicx} % 이미지 삽입
\DeclareGraphicsExtensions{.pdf, .png, .jpg} % 그림 확장자 생략
\usepackage{subcaption} % 이미지 묶어서 삽입하는 subfigure 사용

% 페이지
\usepackage{fancyhdr} % 머리말, 꼬리말
\setlength{\headheight}{18pt} % 헤더 높이 설정
\usepackage{xr-hyper} % 안쓰는 chapter에서 식 참조(aux 파일)할 때 사용
\usepackage{import} % main이 chapter 호출, cahpter가 section 호출(main에 대해 상대경로 사용)

% 캡션, 링크
\usepackage{aliascnt} % 고유 참조 이름(equation, figure 등을 커스텀)
\usepackage{caption} % 캡션 스타일 조정

%% 수식
\usepackage{calc} % 너비, 간격 등 계산
\usepackage{amsmath, amssymb, mathtools} % 본문에서 수학 식 사용
\usepackage{siunitx} % 단위 표기

% 표, 디자인
\usepackage{float} % 표/그림 위치 고정
\usepackage{booktabs} % 표 상단선 등 고급 표 기능
\usepackage{longtable} % 페이지 넘어가는 긴 표
\usepackage{tabularx} % 너비 자동 조정 표
\usepackage{multirow} % 표에서 여러 행 병합
\usepackage{tikz} % 사각박스, 선긋기(교사용 메모 박스)
\usetikzlibrary{calc} % tikz에서 cal 사용
\usepackage{tikzpagenodes} % 사각박스, 선긋기(목차로 이동하는 박스)










\begin{document}


% Calculus, Metric Edition - Chapter 11.3: The Integral Test and Estimates of Sums

This summary covers the key concepts and examples from Section 11.3 of the textbook.

% The Integral Test

The Integral Test provides a method to test an infinite series for convergence by comparing it with an improper integral.

**The Integral Test**
Suppose $f$ is a continuous, positive, decreasing function on $[1, \infty)$ and let $a_n = f(n)$. Then the series $\sum_{n=1}^{\infty} a_n$ is convergent if and only if the improper integral $\int_1^\infty f(x) dx$ is convergent. In other words:
(i) If $\int_1^\infty f(x) dx$ is convergent, then $\sum_{n=1}^{\infty} a_n$ is convergent.
(ii) If $\int_1^\infty f(x) dx$ is divergent, then $\sum_{n=1}^{\infty} a_n$ is divergent.

**Note:** It is not necessary that $f$ be always decreasing. It is sufficient if $f$ is ultimately decreasing, i.e., decreasing for $x$ larger than some number $N$.

% EXAMPLE 1: Using the Integral Test

Test the series $\sum_{n=1}^{\infty} \frac{1}{n^2 + 1}$ for convergence or divergence.

**SOLUTION:**
The function $f(x) = \frac{1}{x^2 + 1}$ is continuous, positive, and decreasing on $[1, \infty)$. We evaluate the integral:
$$
\int_1^\infty \frac{1}{x^2 + 1} dx = \lim_{t \to \infty} \int_1^t \frac{1}{x^2 + 1} dx = \lim_{t \to \infty} [\tan^{-1}x]_1^t
$$
$$
= \lim_{t \to \infty} (\tan^{-1}t - \frac{\pi}{4}) = \frac{\pi}{2} - \frac{\pi}{4} = \frac{\pi}{4}
$$
Since the integral is convergent, the series is convergent by the Integral Test.

% The p-series

The series $\sum_{n=1}^{\infty} \frac{1}{n^p}$ is called the **p-series**.

**1. Convergence of a p-series**
The p-series $\sum_{n=1}^{\infty} \frac{1}{n^p}$ is convergent if $p > 1$ and divergent if $p \le 1$.

% EXAMPLE 2 & 3: Examples of p-series

(a) The series $\sum_{n=1}^{\infty} \frac{1}{n^3}$ is convergent because it is a p-series with $p=3 > 1$.
(b) The series $\sum_{n=1}^{\infty} \frac{1}{n^{1/3}}$ is divergent because it is a p-series with $p=1/3 < 1$.

% EXAMPLE 4: A Divergent Series

Determine whether the series $\sum_{n=1}^{\infty} \frac{\ln n}{n}$ converges or diverges.

**SOLUTION:**
The function $f(x) = \frac{\ln x}{x}$ is positive and continuous for $x > 1$. To check if it's decreasing, we find the derivative:
$$
f'(x) = \frac{(1/x)x - \ln x}{x^2} = \frac{1 - \ln x}{x^2}
$$
$f'(x) < 0$ when $\ln x > 1$, or $x > e$. Thus, $f$ is decreasing for $x > e$. We can apply the Integral Test:
$$
\int_1^\infty \frac{\ln x}{x} dx = \lim_{t \to \infty} \left[ \frac{(\ln x)^2}{2} \right]_1^t = \lim_{t \to \infty} \frac{(\ln t)^2}{2} = \infty
$$
The integral is divergent, so the series is divergent.

---

% Estimating the Sum of a Series

For a convergent series $\sum a_n$, we can approximate its sum $s$ with a partial sum $s_n$. The error in this approximation is the remainder, $R_n = s - s_n$.

**2. Remainder Estimate for the Integral Test**
Suppose $f(k) = a_k$, where $f$ is a continuous, positive, decreasing function for $x \ge n$ and $\sum a_n$ is convergent. If $R_n = s - s_n$, then
$$
\int_{n+1}^\infty f(x) dx \le R_n \le \int_n^\infty f(x) dx
$$

% EXAMPLE 5: Estimating a Sum and Error

(a) Approximate the sum of the series $\sum \frac{1}{n^3}$ by using the sum of the first 10 terms. Estimate the error.
(b) How many terms are required to ensure the sum is accurate to within 0.0005?

**SOLUTION:**
First, we compute the integral: $\int_n^\infty \frac{1}{x^3} dx = \frac{1}{2n^2}$.

(a) The sum of the first 10 terms is $s_{10} \approx 1.1975$. The remainder estimate gives the error:
$$
R_{10} \le \int_{10}^\infty \frac{1}{x^3} dx = \frac{1}{2(10)^2} = \frac{1}{200} = 0.005
$$
The error is at most 0.005.

(b) We need $R_n < 0.0005$. Since $R_n \le \frac{1}{2n^2}$, we set:
$$
\frac{1}{2n^2} < 0.0005 \implies n^2 > \frac{1}{0.001} = 1000 \implies n > \sqrt{1000} \approx 31.6
$$
We need 32 terms to ensure accuracy to within 0.0005.

We can also get a better estimate for the sum $s$ using the following inequality:
**3.** $$
s_n + \int_{n+1}^\infty f(x) dx \le s \le s_n + \int_n^\infty f(x) dx
$$

% EXAMPLE 6: An Improved Estimate

Use the inequality above with $n=10$ to estimate the sum of the series $\sum \frac{1}{n^3}$.

**SOLUTION:**
Using $s_{10} \approx 1.197532$, we have:
$$
s_{10} + \int_{11}^\infty \frac{1}{x^3} dx \le s \le s_{10} + \int_{10}^\infty \frac{1}{x^3} dx
$$
$$
1.197532 + \frac{1}{2(11)^2} \le s \le 1.197532 + \frac{1}{2(10)^2}
$$
$$
1.197532 + 0.004132 \le s \le 1.197532 + 0.005
$$
$$
1.201664 \le s \le 1.202532
$$
If we approximate $s$ by the midpoint of this interval, $s \approx 1.2021$, the error is less than 0.0005.

\end{document}