%%  chapter  %%%%%%%%%%%%%%%%%%%%%%%%%%%
%%%%  section  %%%%%%%%%%%%%%%%%%%%%%%%%
%%%%%%  subsection  %%%%%%%%%%%%%%%%%%%%
%%%%%%%%  subsubection  %%%%%%%%%%%%%%%%

%%  chapter  %%%%%%%%%%%%%%%%%%%%%%%%%%%
\chapter
{Sample Chapter}

%%%%  section  %%%%%%%%%%%%%%%%%%%%%%%%%
\section
{About This Document}
\memo{바로 글 내용을 쓰는 경우만 메모 활용이 가능함.
Section 단독으로 쓰는 경우는 빈줄 삽입됨. \\
아무 말(본문) 없이 메모 쓰면 자리 차지하는 것. 아래 참조}%
\ 이 chapter는 \hlyb{여러 예시}를 담은 곳이다.\\
% 참고문헌 사용하고 싶으면 아래 글을 주석 해제
%\ 이 글은 『Fundamentals of Physics』\cite{halliday12f}, 『Principles of Physics』\cite{halliday12p},『일반물리학』\cite{halliday12p_k}을 참고한 것이다. 이는 일반물리 교재들\cite{halliday12f,halliday12p,halliday12p_k}과 같이 동시에 참조할 수도 있다. \\

%%%%%%  subsection  %%%%%%%%%%%%%%%%%%%%
\subsection{모드 전환}
\memo{챕터: 예시용 샘플 챕터 \\ 섹션: 이 문서에 대한 설명 \\\\ 메모는 본문에만 넣어야 편집 상에 간섭이 없습니다.
줄이 길어지더라도 알아서 메모는 줄바꿈을 하게 됩니다. \\
메모 뒤에는 반드시 `\% 표시'를 해야 의도치 않은 띄어쓰기를 안할 수 있습니다.}%
\ 이 책은 기본적으로 \hlyb{네 가지 모드}가 존재하며, 이를 1부터 4까지의 숫자로 구분한다.
seolfa 함수를 이용하여 모드를 결정하며 1번 모드는 출판용(괄호 없이 전체 나열), 
\hlr{2번 모드}는 학습용(괄호와 빈칸),
3번 모드는 정답 배포용(채워진 괄호),
4번 모드는 교사용으로 (채워진 괄호와 메모)이다. 이를 \seolight{정리하면} 다음과 같다.

\begin{itemize}
  \item 모드 1 $\Rightarrow$ \seolfa{ 괄호 없는 출판용 }
  \item 모드 2 $\Rightarrow$ \seolfa{ 괄호와 빈칸으로 채워진 학습용 } \\
  \bul 모드 3 $\Rightarrow$ \seolfa{ 괄호가 채워진 답안배포용 } \\
  \bul 모드 4 $\Rightarrow$ \seolfa{ 괄호가 채워지고 메모가 노출되는 교사용 }
\end{itemize}

\bul 글머리 앞의 작은 기호를 bul로 만들었음 \\
기존의 item이나 bullet은 크기가 크고 자리를 많이 차지함.
itemize 역시 틀의 변형이 필요하여 box 및 bullet을 새로 만들었음.

이에 반해, seolb는 seola와 기능이 비슷하나 괄호가 없다. 표 안의 내용 등에 사용된다.

\begin{graybox}
  \bul 모드 1 $\longrightarrow$ \seolfb{ 출판용 } \\
  \bul 모드 2 $\longrightarrow$ \seolfb{ 빈칸으로 채워진 학습용 } \\
  \bul 모드 3 $\longrightarrow$ \seolfb{ 내용이 채워진 답안배포용 } \\
  \bul 모드 4 $\longrightarrow$ \seolfb{ 내용이 채워지고 메모가 노출되는 교사}용
  테스트테스트테스트\seolfb{테스트테스트테스트테스트테스트}테스트테스트테스트테스트
\end{graybox}



%%%%%%  subsection  %%%%%%%%%%%%%%%%%%%%
\subsection{페이지 전환}
\memo{페이지 전환 종류와 방법 설명함}%
\ 페이지를 넘길 땐 \blue{newpage}와 \red{clearpage}가 있는데,
\blue{newpage}는 페이지를 새로 생성할 뿐이므로, 그림이나 표가 다음장으로 넘어갈 수 있다.
이에 반해, \red{clearpage}는 해당 페이지를 기점으로 새로운 페이지가 생성되므로, 확고한 페이지 나눔을 사용할 수 있다.
따라서, \seolight{\red{clearpage}}로 다음 페이지로 넘어가도록 하겠다.
\clearpage


%%%%  section  %%%%%%%%%%%%%%%%%%%%%%%%%
\section
{Text Formatting}

%%%%%%  subsection  %%%%%%%%%%%%%%%%%%%%
\subsection{글자 모양}
\memo{섹션: 글자 서식 \\ 여러 가지 글자 모양에 대한 설명}%
\ 사전에 지정한 매크로를 이용하여 글자색을 나타낼 수 있고, 언제든 .sty에 추가할 수 있다.
예를 들면, \blue{블루 blue}, \red{레드 red}, \gray{그레이 gray}, \blueb{블루볼드 blueb} \redb{레드볼드 redb} \grayb{그레이볼드 grayb}
와 같이 나타낼 수 있고, 볼드체와 이탤릭체는 bd, itl으로 나타낼 수 있다.(폰트가 지원하지 않을 수 있으므로 생략한다.) \\
\ 형광펜을 사용하는 방법은 \hly{형광펜 하이라이트 hly} \hlyb{형관펜 볼드 hlyb}이다. \hlr{빨간색 형광펜 hlr, hlrb}도 사용할 수 있다. \\
\ 교사용 하이라이트는 \seolight{교사용 seolight}를 사용한다. 모드 4에서만 등장한다.\\
모드 \solight{3,4에서 나오는 solight}도 있고, 이는 답안용에 해당한다. 

%%%%%%  subsection  %%%%%%%%%%%%%%%%%%%%
\subsection{자동 넘버링}
\memo{an, bn 설명}%
\an\ an은 자동으로 숫자가 증가하는 기능이다. \\
\an\ 다음 내용을 이렇게 쓰고 \\
중간에 내용을 적고, 식을 쓰다가 다시 쓰면 이어진다. \\
\an\ 다시 썼다가, 초기화하고 \\
\anset
\an\ 다시 쓰고 \\
\textbf{\an\ 굵게 지정해서 다시 쓰고} \\
중간에 내용 적었다가 옆으로도 \an\ 다시 쓰고 \\
\bn\ bn은 이렇게 쓰고 \\
\bn\ 다시 쓰고 \\
\bnset
초기화 했다가 \\
\bn\ 다시 쓸 수 있다. \\
\biglabel{12.} 빅라벨(big label) 사용 방법 \\
\biglabel{\an} an도 빅라벨 사용 가능함 \\
\biglabel{가.} 한글 빅라벨도 가능함. 그냥 큰 글씨인 셈 \\

%%%%%%  subsection  %%%%%%%%%%%%%%%%%%%%
\subsection{글자 배치}
\memo{글자 정렬 설명}%
\centertext{여러 줄을 \\ 가운데 정렬하기 위하여 centertext를 사용할 수 있다.\\
기타 자세한 사항은 example article에 상세히 나와있으니 참고하면 된다.}

\clearpage

\insertTeacherPages % {페이지1}{페이지2}
{ % 교사용 페이지 1 시작
이곳은 교사 전용 페이지 1입니다. \\
insertTeacherPages\{페이지1\}\{페이지2\}로 삽입할 수 있고,
페이지는 양면이며, seolmode가 3인 경우에만 삽입됩니다. \\
또한 페이지번호가 무시되어 mode에 따라 전체 틀에 영향을 주지 않습니다.
} % 교사용 페이지 1 끝
{ % 교사용 페이지 2 시작
이곳은 교사 전용 페이지 2 입니다. \\
반드시 2페이지(양면)을 사용해야만 편집에 영향을 주지 않습니다.
} % 교사용 페이지 2 끝


%%%%  section  %%%%%%%%%%%%%%%%%%%%%%%%%
\section
{Box And Table}

%%%%%%  subsection  %%%%%%%%%%%%%%%%%%%%
\subsection{Box}
\memo{섹션: 박스와 표 \\ 서브섹션: 박스 종류와 기능}%
글자만 \grayinline{박스로} 채우기는 잘 안쓴다. 형광펜을 쓰는 게 더 나아 보임\\
\begin{graybox}
회색 박스
\end{graybox}

\begin{redbox}
빨간배경 박스 \\
redbox
\end{redbox}

\begin{bluebox}
파란 배경 박스 \\
bluebox
\end{bluebox}

\begin{eqbox}
F = ma ~~~~~ \text{(수식용 eqbox도 있음. equation과 같음)}
\end{eqbox}
\vspace{10pt}%

\begin{eqbox}
a = \frac{F}{m} ~~~~~ \text{\parbox{5.4cm}{(eqbox는 기본 형태가 수식이라, \\\hspace*{1pt} 한글 입력시 text 함수를 써야 한다.)}}
\end{eqbox}
\vspace{10pt}%

\begin{checkbox}
확인문제용 박스(넘버링 됨)도 있다. 본문과 같이 기본 형태가 text이다. \\
번호는 section 단위로 리셋된다. \\
수식을 쓰려면 \$을 이용하여 $F=ma$ 와 같이 나타낼 수 있다. \\
\$을 두개 쓰면 줄바꿈, 가운데 정렬이 되므로, 본문 중간에 넣지 않도록 주의한다.
\end{checkbox}

\begin{checkbox*}
*을 붙이면 넘버링이 되지 않는다
\end{checkbox*}

\begin{practicebox}{여기에 제목을 추가 입력}
연습문제용 박스도 있음
\end{practicebox}

\begin{practicebox*}{여기에 제목을 추가 입력}
*을 붙이면 넘버링이 되지 않는다
\end{practicebox*}

\begin{practicebox}{넘버링은 자동으로 됨}
연습문제\\
풀어보세요
\end{practicebox}

%%%%%%  subsection  %%%%%%%%%%%%%%%%%%%%
\subsection{Table}
\memo{표를 작성하는 방법}%
간단한 표는 본문 내부에 삽입하지만, 내용이 복잡해지므로 table에 모아두고 참조할 수 있다. \\
최종 결과는 나중에 보고, 일단 1단계부터 살펴보자. \\

tabular 함수는 표 자체를 만드는 것이고,
\hly{캡션, 여백까지 모두 갖으려면 table}로 tabular를 감싸야 한다.
따라서, \hlyb{처음부터 table에 tabular를 넣은 함수를 지정하고 쓰는 것}이 낫다.
아래는 tabular만을 사용한 예시로, LCR(좌중우)로 정렬하였다.
표 자체는 기본적으로 왼쪽정렬 된다. \\
\vspace{-1em}% \\ % 이정도 여백 있어야 적당히 보이는 것 같음
\begin{tabular}{|L{3cm}|C{3cm}|R{2cm}|} % 열 3개: 폭 3, 3, 5cm 정렬은 LCR, 테두리 있음
\hline
\graycell{구분} & \bluecell{내용} & \bluecell{내용} \\
\hline
\redcell{경고} & 중요 사항 & \\
\hline
\end{tabular}%

\vspace{10pt}%
이번에도 tabular만 쓴 예시이고, p형태로 셀마다 정렬을 따로하였다.
표 자체를 가운데 정렬하였다. \\
\vspace{-2.1em}% % 가운데 정렬시 이정도 여백
\begin{center}
\begin{tabular}{|p{4cm}|p{4cm}|}
\hline
\raggedright 왼쪽 정렬 & \centering 가운데 정렬 \tabularnewline
\hline
\raggedleft 오른쪽 정렬 & \raggedright 다시 왼쪽 정렬 \tabularnewline
\hline
\end{tabular}
\end{center}%

\vspace{10pt}%
이번엔 table의 요소를 결합한 예시이다. \\
표에 label을 달기 위해서는 table 안에 tabular를 넣고, caption, label을 쓴다.
아래 예시에서 캡션을 아래에 썼지만, tabular 위에 쓰면 캡션은 위로 간다.
\begin{table}[h] % talbe 환경 시작, here
  \centering % 가운데 정렬
  \begin{tabular}{|c|c|}
    \hline
    항목 & 내용 \\
    \hline
    A & 설명 \\
    \hline
  \end{tabular}
  \caption{캡션. 번호는 chapter에 맞게 부여된다.}
  % \label{tab:example} % 예시이므로 라벨은 따로 만들지 않음.
\end{table}%

\vspace{10pt}%
애초에 table의 요소를 모두 넣어서 \hlyb{tab함수}를 만들었다.
표는 언제나 가운데정렬되며, 예시이므로 참조용 라벨은 비웠다. 캡션은 표 위에 작성된다.

\tab{학생 성적 비교}{}{c c c}{
이름 & 점수 & 등급 \\
홍길동 & 95 & A \\
김철수 & 88 & B
}

\tab{학생 성적 비교}{}{c | c c}{
이름 & 점수 & 등급 \\
\hline
홍길동 & 95 & A \\
김철수 & 88 & B
}

\vspace{10pt}%
치수를 3cm 등으로 작성하면, 셀의 너비를 지정할 수 있다. 또한 선으로 감쌀 수 있다.
\tab{전체를 선으로 감싼 표}{}{|L{3cm}|C{3cm}|R{2cm}|}{
\graycell{구분} & \bluecell{내용} & \bluecell{내용} \\
\hline
\rightcell{오른쪽 정렬} & \leftcell{다시 왼쪽 정렬} &
}
% 길이 지정은 p,m,b 로만 가능함. 위의 L, C, R은 이미 함수로 지정해놔서 되는 것
% 즉, |C{4cm}|는 되는데, |C{4cm}|는 안된다.
% R, L, C는 길이를 지정해서 쓰기 위한 함수로, 새로 만든 함수이다. 
\tab{다른 제목}{}{|p{4cm}|p{4cm}|}{
\leftcell{왼쪽 정렬} & 기본 정렬 \\
\hline
\rightcell{오른쪽 정렬} & \leftcell{다시 왼쪽 정렬}
}

\tab{}{}{c}{
캡션도 라벨도 없는 \\
단순한 표는 다음과 같이 만들 수 있음
}

\tab{복잡한 표의 예시}{}{| L{3cm} | c | C{2cm} |}{ % 수치를 안쓰면 대문자 아닌 소문자 c 쓴다. 자동 너비맞춤
\graycell{구분} & \redcell{내용} & \bluecell{내용} \\
\hline
\leftcell{왼쪽} & \centercell{가운데} & \rightcell{오른쪽} \\
\hline
\bluecell{\leftcell{왼쪽}} & \graycell{\centercell{가운데}} & \redcell{\rightcell{오른쪽}}
}

\vspace{10pt}%
더 나아가서, 표 자체를 선언할 수 있도록 DeclareTable 함수를 만들고,
tables.tex에 모아두고 참조하여 사용한다. Figure도 마찬가지임.

%% 변환열 예시. 일반물리학 교재에서 가져옴. 라벨은 비워두었음
\DeclareTable
  {}
  {\exampletable}
  {Some Heats of Transformation}
  {l S[table-format=4.1] S[table-format=3.1] S[table-format=4.1] S[table-format=4.0]}{
          & \multicolumn{2}{c}{Melting} & \multicolumn{2}{c}{Boiling}  \\
          \cmidrule(lr){2-3} \cmidrule(lr){4-5}
          & {Melting} & {Heat of} & {Boiling} & {Heat of} \\
          & {Point} & {Fusion} & {Point} & {Vaporization} \\
Substance & {($\mathrm{K}$)} & {$L_f\,(\mathrm{kJ/kg})$} & {($\mathrm{K}$)} & {$L_v\,(\mathrm{kJ/kg})$} \\
\hline
Hydrogen & 14.0  & 58.0  & 20.3   & 455  \\
Oxygen   & 54.8  & 13.9  & 90.2   & 213  \\
Mercury  & 234   & 11.4  & 630    & 296  \\
Water    & 273   & 333   & 373    & 2256 \\
Lead     & 601   & 23.2  & 2027   & 858  \\
Silver   & 1235  & 105   & 2323   & 2336 \\
Copper   & 1356  & 207   & 2868   & 4730}

미리 표를 선언해두고 명령어로 깔끔하게 불러올 수 있음 \\
\exampletable
여기서, \hly{S는 소수점 자리를 일치}시키는 것으로, S만 써도 됨 \\
하지만 소수점이 중심으로 가도록 정렬되므로, 정수만 있다면 정렬이 예쁘지 않음 \\
table-format=4.1은 \hly{정수 4자리 + 소수점 + 소수 1자리}로 이루어진 배열이고, \\
가장 마지막 열은 정수만 있으므로 4.0(또는 4)으로 작성하여 정렬하였음. \\
데이터의 형태를 보고 맞추면 정렬이 예쁘게 됨. \\
이때, \hly{숫자가 아닌 모든 텍스트(단위 포함)는 중괄호로} 묶어줘야 함. \\

아래는 다른 예시로 음수와 \approx 기호를 포함하고 있음. 포맷에 넣어줘야 자리수 정렬 가능함 \\
\DeclareTable
  {}
  {\exampletabletwo}
  {Some corresponding temperatures}
  {l S[table-format=\approx-3.1] S[table-format=-3.1]}
  {Temperature & {$^\circ\mathrm{C}$} & {$^\circ\mathrm{F}$} \\\hline
    Boiling point of water & 100 & 212 \\
    Normal body temperature & 37.0 & 98.6 \\
    Accepted comfort level & 20 & 68 \\
    Freezing point of water & 0 & 32 \\
    Zero of Fahrenheit scale & \approx-18 & 0 \\
    Scales coincide & -40 & -40}%
명령어에 숫자 안들어감..\\
\exampletabletwo

\DeclareTable
  {}
  {\exampletablethree}
  {The First Law of Thermodynamics: Four Special Cases}
  {l >{$}r<{$}@{${}={}$}>{$}l<{$} >{$}r<{$}@{${}={}$}>{$}l<{$}}{
  \multicolumn{5}{c}{The Law : $\Delta E_{\mathrm{int}} = Q - W$ (\autoref{eq:first_law_of_thermodynamics})} \\
  \hline
  Process & \multicolumn{2}{c}{Restriction} & \multicolumn{2}{c}{Consequence} \\
  \hline
  Adiabatic       & Q & 0             & \Delta E_{\mathrm{int}} & -W \\
  Constant volume & W & 0             & \Delta E_{\mathrm{int}} & Q  \\
  Closed cycle    & \Delta E_{\mathrm{int}} & 0  & Q & W \\
  Free expansion  & Q & W = 0 & \Delta E_{\mathrm{int}} & 0}
% >{$}r<{$}: 수학 모드로 오른쪽 정렬
% @{${}={}$}: 가운데 등호 출력 (간격 없이)
% 두 쌍씩 반복해서 왼쪽=오른쪽 형태 정렬을 두 번 반복함
등호에 정렬하는 예시 \\
\exampletablethree
