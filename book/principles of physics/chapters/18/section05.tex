%%%%  section  %%%%%%%%%%%%%%%%%%%%%%%%%
\section{The First Law of Thermodynamics}
\memo{section: 열역학 제1법칙}%
\ 본문 내용

%%%%%%  subsection  %%%%%%%%%%%%%%%%%%%%
\subsection{A Closer Look at Heat and Work}
\memo{sub: 열과 일에 대한 자세한 고찰}%
\ 본문 내용

%% Fig. 피스톤 속 기체가 하는 일과 열 전달
\gasinapistonandheattransfer
\memo{피스톤이 달린 원통 안에 기체가 갇혀 있다. 열저장고의 온도 $T$를 조절하여
열 $Q$를 기체에 더하거나 기체에서 제거할 수 있다. 피스톤이 움직일 때 기체는 $W$의 일을한다.}%
\autoref{fig:gas_in_a_piston_and_heat_transfer}\은
피스톤 속 기체가 하는 일과 열 전달을 나타낸 것이다.

\begin{align}
    dW&= \vec{F} \cdot d\vec{s} = (pA)(ds) = p (A\, ds) \notag \\
    &= p\, dV
\end{align}

%% eq box. 기체가 한 일
\begin{eqbox} W = \int dW = \int_{V_i}^{V_f} p\, dV
\label{eq:work_done_by_a_gas} \end{eqbox}

%% Fig. 초기 상태에서 최종 상태로 이동할 때 시스템(기체)이 수행하는 일의 양
\workinastatechange
\memo{(a) 색칠한 부분은 초기상태 $i$에서 최종상태 $f$로 가면서 계가 한 일 $W$를 나타낸다.
계의 부피가 증가하므로 일 $W$는 양의 값이다. \\
(b) $W$ 역시 양의 값이지만 (a)의 경우보다 크다. \\
(c) $W$는 여전히 양의 값이지만 (a)보다 더 작다. \\
(d) $W$는 훨씬 더 작을 수도 있고 (경로 $icdf$), 훨씬 더 클 수도 있다(경로 $ighf$). \\
(e) 상태 $f$에서 $i$로 갈 때 외부력이 가해져서 기체가 수축된다. 이때 계(기체)가 한 일 $W$는 음의 값이다. \\
(f) 한 번의 순환과정 동안 계가 한 알짜 일 $W_{\mathrm{net}}$은 색칠한 부분의 면적이다.}%
\autoref{fig:work_in_a_state_change}\은 초기 상태에서 최종 상태로 이동할 때 시스템(기체)이 수행하는 일의 양

\begin{checkbox}
The $p-V$ diagram here shows six curved paths (connected by vertical paths)
that can be followed by a gas. Which two of the curved paths should
be part of a closed cycle (those curved paths plus connecting vertical paths)
if the net work done by the gas during the cycle is to be at its maximum positive value? \\
\memo{그림과 같이 기체의 변화를 나타내는 수직경로로 연결되어 있는 여섯 가지 $p-V$ 곡선이 있다.
순환과정 동안 기체가 한 알짜일이 최대가 되도록 닫힌 경로를 만들때,
곡선 가운데 어느 두 개를 골라야 하는가?}%

%% Fig. 여섯 가지 pv 곡선
\quickfig{six_thermodynamic_paths_on_a_pv_diagram}
\end{checkbox}

\begin{solbox}
\bnset
\bd{Sol: } \\
\bn  \\
\bn  \\
\bn  \\
\hspace*{1em} (ㅇㅇ \\
\hspace*{1em} \, ㅇㅇ) \\

\bd{Ans(a): } \\
 \\
\bd{Ans(b): } \\

\end{solbox}

%%%%%%  subsection  %%%%%%%%%%%%%%%%%%%%
\subsection{First Law of Thermodynamics}
\memo{sub: 열역학 제1법칙}%
\ 본문 내용

%% eq box. 열역학 제1법칙
\begin{eqbox} \Delta E_{\text{int}} = E_{\text{int},f} - E_{\text{int},i} = Q - W ~~~~~ \text{(first law)}
\label{eq:first_law_of_thermodynamics} \end{eqbox}

%% eq box. 열역학 제1법칙 미분형
\begin{eqbox} dE_{\text{int}} = dQ - dW ~~~~~ \text{(first law)}
\label{eq:differential_form_of_first_law_of_thermodynamics} \end{eqbox}

\begin{checkbox}
The figure here shows four paths on a $p-V$ diagram along which a gas can be
taken from state $i$ to state $f$. Rank the paths according to
(a) the change $\Delta E_{\mathrm{int}}$ in the internal energy of the gas,
(b) the work $W$ done by the gas, and (c) the magnitude of the energy
transferred as heat $Q$ between the gas and its environment, greatest first. \\
\memo{그림의 $p-V$ 곡선 네 가지는 기체가 초기상태 $i$에서 최종상태 $f$로 변할 때의 경로이다.
다음의 양이 큰 경로부터 순서대로 나열하여라 \\
(a) 기체 내부에너지의 변화 $\Delta E_{\mathrm{int}}$, \\
(b) 기체가 한 일 $W$, \\
(c) 기체와 주위 사이에 전달된 열에너지 $Q$의 크기.}%

%% Fig. 네 가지 pv 곡선
\quickfig{four_thermodynamic_paths_on_a_pv_diagram}
\end{checkbox}

\begin{solbox}
\bnset
\bd{Sol: } \\
\bn  \\
\bn  \\
\bn  \\
\hspace*{1em} (ㅇㅇ \\
\hspace*{1em} \, ㅇㅇ) \\

\bd{Ans(a): } \\
 \\
\bd{Ans(b): } \\

\end{solbox}

%%%%%%  subsection  %%%%%%%%%%%%%%%%%%%%
\subsection{Some Special Cases of the First Law of Thermodynamics}
\memo{sub: 열역학 제1법칙의 특수한 경우}%
\ 본문 내용

%% Tab. 열역학 제1법칙: 네 가지 특별한 과정
\thefirstlawofthermodynamics
\memo{단열, 등적, 순환, 자유팽창}%
\autoref{tab:the_first_law_of_thermodynamics}\은 열역학 제1법칙의 네 가지 특별한 과정을 나타낸 것이다. \\
\\
\anset
\bd{\an \itl{Adiabatic processes.}} 단열과정은
\begin{equation} \Delta E_{\text{int}} = -W ~~~~~ \text{(adiabatic process)} \end{equation}

%% Fig. 단열팽창
\adiabaticexpansion
\memo{단열팽창은 피스톤 위에 놓인 납알을 천천히 없애면 구현할 수 있다.
납알을 더하면 언제든지 반대 과정을 만들 수 있다.}%
\autoref{fig:adiabatic_expansion}\은 단열팽창에 대한 

\bd{\an \itl{Constant-volume processes.}} 등적과정은
\begin{equation} \Delta E_{\text{int}} = Q ~~~~~ \text{(constant-volume process)} \end{equation}

\bd{\an \itl{Cyclical processes.}} 순환과정은
\begin{equation} Q = W ~~~~~ \text{(cyclical process)} \end{equation}

\bd{\an \itl{Free expansions.}} 자유팽창은
\begin{equation} \Delta E_{\text{int}} = 0 ~~~~~ \text{(free expansion)} \end{equation}

%% Fig. 자유팽창
\freeexpansion
\memo{자유팽창과정의 초기단계. 잠금마개를 열면 기체가 양쪽 방을 채우면서,
결국에는 평형상태에 도달하게 된다.}%
\autoref{fig:free_expansion}\은 자유팽창과정의 초기 단계를 나타낸 것이다.

\begin{checkbox}
For one complete cycle as shown in the $p-V$ diagram here, are
(a) $\Delta E_{\text{int}}$ for the gas and
(b) the net energy transferred as heat $Q$ positive, negative, or zero? \\
\memo{오른편의 $p-V$ 그림은 어떤 순환과정이다. \\
(a) 기체의 $\Delta E_{\text{int}}$, \\
(b) 알짜 열에너지 전달량 $Q$는 양수인가, 음수인가, 아니면 $0$인가?}%

%% Fig. pv 순환과정의 예시
\quickfig{example_of_a_pv_cycle}
\end{checkbox}

\begin{solbox}
\bnset
\bd{Sol: } \\
\bn  \\
\bn  \\
\bn  \\
\hspace*{1em} (ㅇㅇ \\
\hspace*{1em} \, ㅇㅇ) \\

\bd{Ans(a): } \\
 \\
\bd{Ans(b): } \\

\end{solbox}

\begin{practicebox}{First law of thermodynamics: work, heat, internal energy change}
%% Fig. 일정한 압력에서 끓는 물
\boilingwateratconstantpressure
\memo{물이 일정한 압력에서 끓는다. 액체인 물이 완전히 수증기로 변환할 때까지
열저장고는 에너지를 열로 전달한다. 팽창하는 기체는 피스톤을 밀어 올리면서 일을 한다.}%
Let $1.00\,\mathrm{kg}$ of liquid water at $100\,^\circ\mathrm{C}$
be converted to steam
at $100\,^\circ\mathrm{C}$ by boiling at standard atmospheric pressure
(which is $1.00\,\mathrm{atm}$ or $1.01 \times 10^{5}\,\mathrm{Pa}$)
in the arrangement of \autoref{fig:boiling_water_at_constant_pressure}.
The volume of that water changes from an initial value of
$1.00 \times 10^{-3}\,\mathrm{m^3}$ as a liquid to
$1.671\,\mathrm{m^3}$ as steam. \\
(a) How much work is done by the system during this process? \\
(b) How much energy is transferred as heat during the process? \\
\memo{원서에서는 $1.00\,\mathrm{kg}$인데, \hly{번역서는 $5.00\,\mathrm{kg}$}로 나옴 \\
열역학 제1법칙: 일, 열, 내부에너지 변화 \\
\autoref{fig:boiling_water_at_constant_pressure}처럼
$100\,^\circ\mathrm{C}$ 물 $1.00\,\mathrm{kg}$이 표준대기압
($1.00\,\mathrm{atm}$, 즉 $1.01 \times 10^{5}\,\mathrm{Pa}$)에서
$100\,^\circ\mathrm{C}$의 수증기로 바뀌었다. 물의 초기 부피는
$1.00 \times 10^{-3}\,\mathrm{m^3}$였고 수증기가 되었을 때 부피는
$1.671\,\mathrm{m^3}$였다. \\
(a) 계가 한 일은 얼마인가? \\
(b) 이 과정에서 열의 형태로 전달된 에너지는 얼마인가?}%
Sol: \\
Ans:
\end{practicebox}
\clearpage