%%%%  section  %%%%%%%%%%%%%%%%%%%%%%%%%
\section{The Celsius and Fahrenheit Scales}
\memo{section: 섭씨온도와 화씨온도}%
\ 여러 온도 척도를 비교하고 관계를 찾아보자.

%%%%%%  subsection  %%%%%%%%%%%%%%%%%%%%
\subsection{The Celsius and Fahrenheit Scales}
\memo{sub: 섭씨온도와 화씨온도}%
\ 여러 척도 중 섭씨온도는 도($^\circ$)의 단위로 측정하며
\hly{섭씨 1도는 켈빈 눈금의 1도와 크기가 같다}.
섭씨온도의 0도는 절대영도와 다르다. 두 온도척도는 다음의 관계를 가진다.
\begin{equation} T_C = T - 273.15\,^\circ \end{equation}

\ 화씨온도는 미국에서만 사용하는데, 섭씨온도와 화씨온도 사이의 관계는 다음과 같다.
\begin{equation} T_F = \frac{9}{5} T_C + 32\,^\circ \end{equation}

\ 몇 가지 온도의 값에 대한 비교는 다음 표와 같다. \\

%% Tab. 몇 가지 온도의 값
\somecorrespondingtemperatures
\memo{Strictly, the boiling point of water on the Celsius scale is
$99.975\,^\circ\mathrm{C}$,
and the freezing point is $0.00\,^\circ\mathrm{C}$. Thus, there is slightly lessthan
$100\,^\circ\mathrm{C}$ between those two points.\\
엄밀히 말하면 섭씨로 물의 끓는점은 \seolight{$99.975\,^\circ\mathrm{C}$}이고,
어는점은 \seolight{$0.00\,^\circ\mathrm{C}$}이다.
따라서 두 점의 차이는 $100\,^\circ\mathrm{C}$보다 \seolight{약간 작다}. \\
 \\
옛날엔 끓는점이 $100\,^\circ\mathrm{C}$이라고 했지만,
\seolight{새로운 도량형으로 삼중점을 $0.01\,^\circ\mathrm{C}$}로 잡는 등
새로운 온도척도가 정의되면서 \seolight{정밀 측정 결과 $99.974\,^\circ\mathrm{C}$}가 됨 \\
책과도 살짝 다름}%

%% Fig. 켈빈, 섭씨, 화씨온도 눈금 비교
\kelvincelsiusfahrenheittemperaturescalescompared
\memo{켈빈, 섭씨, 화씨온도의 비교. \\
앞에서 이야기한 대로 \seolight{여러 사람이 제안한 척도일 뿐}이다.
\seolight{Celsius}는 물의 끓는점과 어는 점을 \seolight{100, 0}이라는 척도로 제안하였고,
간격을 \seolight{100등분} 하였으며
Fahrenheit는 \seolight{212, 32}로 제안하였고, 간격을 \seolight{180등분} 한 것이다. \\
 \\
최초에 Fahrenheit는 \seolight{염화암모늄+물+얼음 혼합물}의 최저 온도 (소금물 기반 인공 냉각)에서
\seolight{가장 낮은 온도}를 찾으려고 했고, 그것을 \seolight{0으로 잡으려고} 하였다.
또한 사람의 \seolight{체온을 약 96정도}로 잡으려고 하였다. \\
셀시우스, 켈빈, \seolight{뉴턴, 뢰머} 등 온도 척도는 다양하다.}
\autoref{fig:kelvin_celsius_fahrenheit_temperature_scales_compared}\은
켈빈, 섭씨, 화씨온도 눈금을 비교한 것이다.

\ 두 척도는 문자 C와 F로 구별하여 섭씨 $0\,^\circ$가 화씨 $32\,^\circ$와 같음을
다음과 같이 나타낸다.
\begin{equation*} 0\,^\circ\mathrm{C} = 32\,^\circ\mathrm{F} \end{equation*}
%
\ 섭씨로 온도 차이가 5도이면 화씨로 9도 차이가 있음을 다음과 같이 나타낼 수 있다.
이 경우 도를 나타내는 기호는 C 다음에 온다. \\
\memo{$5\,^\circ\mathrm{C}$가 $9\,^\circ\mathrm{F}$라는 뜻이 아니라
\seolight{섭씨 5개의 눈금이 화씨 9개 눈금과 같다는 의미}로,
간격을 비교하기 위해 사용함}%
\begin{equation*} 5\,\mathrm{C}^\circ = 9\,\mathrm{F}^\circ \end{equation*}
\clearpage



\begin{checkbox}
The figure here shows three linear temperature scales with
the \seolight{freezing and boiling} points of water indicated. \\
(a) Rank the degrees on these scales by size, greatest first. \\
(b) Rank the following temperatures, highest
first: $50\,^\circ\mathrm{X}$, $50\,^\circ\mathrm{W}$, and $50\,^\circ\mathrm{Y}$.\\
\memo{오른편 그림은 물의 \seolight{어는점과 끓는점}이 표시된 세 가지 선형 온도척도이다.\\
(a) 온도 \seolight{1도의 차이가 가장 큰 순서}대로 나열하여라. \\
(b) $50\,^\circ\mathrm{X}$, $50\,^\circ\mathrm{W}$, $50\,^\circ\mathrm{Y}$의
온도 중에서 가장 \seolight{높은 순서}대로 나열하여라.}%

%% Fig. 세 가지 선형 온도척도
\quickfig{three_linear_temperature_scales}
\
\end{checkbox}

\begin{solbox}
\bnset
\bd{Sol: } \\
\bn 1도 온도는 같은 두 지점 간격을 몇 등분하느냐를 비교하면 된다. \\
\bns \solight{모두 90등분} 되어있으므로, (a)의 답은 \solight{모두 동일}하다.  \\
\bn 눈금 간격이 같으므로 50도까지 \solight{어느 만큼 이동}하는지 비교한다. \\
\bns 순서대로 맨 위에서 \solight{20칸, 70칸, 40칸} 만큼 아래로 이동하므로 \\
\bns X, Y, W 순서대로 높은 온도를 의미한다.\\
\memo{순서가 x,y,z가 아니라 z,y,w네..}%
\bd{Ans(a): } 모두 동일하다. \\
\bd{Ans(b): } X, Y, W
\end{solbox}

\begin{practicebox}{Conversion between two temperature scales}
Suppose you come across old scientific notes that describe
a temperature scale called $\mathrm{Z}$ on which the boiling point of
water is \seolight{$65.0\,^\circ\mathrm{Z}$} and
the freezing point is \seolight{$-14.0\,^\circ\mathrm{Z}$}. To what
temperature on the \seolight{Fahrenheit scale} would a temperature
of \seolight{$T = -98.0\,^\circ\mathrm{Z}$} correspond?
Assume that the $\mathrm{Z}$ scale is linear; that is,
the size of a $\mathrm{Z}$ degree is the same everywhere on the $\mathrm{Z}$ scale. \\
\memo{두 온도척도의 전환 \\\
 \\
옛날의 과학 원고를 읽다가 온도척도를 $\mathrm{Z}$라고 하여 \seolight{물의 끓는점과
어는점을 각각 $65.0\,^\circ\mathrm{Z}$와 $-14.0\,^\circ\mathrm{Z}$}라고
쓴 기록을 보았다. \seolight{$T = -98.0\,^\circ\mathrm{Z}$는 화씨로 몇 도}인가?
$\mathrm{Z}$라는 척도는 선형, 즉 1도$\mathrm{Z}$는
$\mathrm{Z}$ 척도 어디서나 같다고 가정하자.}%
%% Fig. 화씨온도와 비교한 미지의 온도척도
\unknowntemperaturescale 
\vspace{0pt}%
\end{practicebox}
\addlines{2} % 페이지 크게 함
\begin{solbox}
\bnset
\bd{Sol: } \\
\bn 동일 온도에서 비교하면, \solight{$212\,^\circ\mathrm{F}$는 $65.0\,^\circ\mathrm{Z}$}이고
\solight{$32\,^\circ\mathrm{F}$는 $-14.0\,^\circ\mathrm{Z}$}이다. \\
\bn $Z$ 척도를 $F$ 척도로 바꾸려면 \solight{간격에 대한 보정}은 \\
\bns ($212-32)\,\mathrm{F}^\circ = (65-(-14))\,\mathrm{Z}^\circ$
에서 $180\,\mathrm{F}^\circ = 79\,\mathrm{Z}^\circ$
이므로, \solight{$1\,\mathrm{Z}^\circ = \frac{180}{79}\,\mathrm{F}^\circ$}이다. \\
\bn $Z$ 척도의 \solight{$-14.0\,^\circ\mathrm{Z}$에서 $-84.0\,^\circ\mathrm{Z}$}만큼
온도가 내려가므로, \\
\bns 동일한 온도의 \solight{$32\,^\circ\mathrm{F}$에서
$84 \times \frac{180}{79}\,\mathrm{F}^\circ$}만큼 온도가 내려가면 된다. \\
\bn $32 - 84 \times \frac{180}{79} = -159.39\,(^\circ\mathrm{F})$이다. \\
\bd{Ans: } $-159.39\,^\circ\mathrm{F}$
\end{solbox}
\clearpage