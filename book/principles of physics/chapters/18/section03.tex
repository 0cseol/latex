%%%%  section  %%%%%%%%%%%%%%%%%%%%%%%%%
\section{Thermal Expansion}
\memo{section: 열팽창}%
\ 모든 물체는 온도에 따라 크기가 변한다.

%%%%%%  subsection  %%%%%%%%%%%%%%%%%%%%
\subsection{Thermal Expansion}
\memo{sub: 열팽창}%
\ 온도 상승에 따른 \hly{열팽창}은 주위에서 흔히 볼 수 있다.
\seolight{한여름}에 철로가 팽창하여 휘지 않도록 철로의 \seolight{간격}은 약간 떨어져 있지만,
\seolight{한겨울}에 지나치게 수축하여 기차가 \seolight{이탈할 위험}도 있다.
따라서 정확한 특징 파악과 계산이 필요하다.

%% Fig. 콩코드 비행기
\concorde
\memo{콩코드 비행기(\autoref{fig:concorde})가 음속보다 빨리 날면 공기에 의한 마찰로
열팽창이 일어나서 길이가 \seolight{$12.5\,\mathrm{cm}$나 늘어난다}.
(비행기의 앞부분은 \seolight{$128\,^\circ\mathrm{C}$},
비행기의 꼬리 부분은 $90\,^\circ\mathrm{C}$이고,
객실의 유리창을 만지면 상당히 따뜻하다).}%
\ 콩코드 비행기(\autoref{fig:concorde})는 \seolight{초음속 비행}이 가능하기 때문에
공기에 의한 \seolight{마찰열}이 많이 발생하므로 \seolight{기체의 열팽창}을 고려해야 한다. \\

%% Fig. 바이메탈
\bimetalstrip
\memo{(a) 온도 $T_0$에서 \seolight{황동과 강철}이 용접되어 붙어 있는 금속이중띠.
(b) 이 기준온도보다 온도가 올라가면 띠는 그림처럼 휘어지고,
내려가면 위로 휘게 된다. 많은 검온계가 이 원리로 작동하여
온도의 증감에 따라 전기 접촉을 끊거나 잇는다. \\
 \\
보통 그대로 \seolight{바이메탈}이라고 부른다.}%
\ 금속이중띠(\autoref{fig:bimetal_strip})는 팽창이나 수축하는 길이가 다르기 때문에 휘어진다.
\begin{sssbox}
\ 황동과 강철 중에 온도가 변할 때 더 많이 변하는 것은 \seolfa{황동}이다.
잘 변하고 변하지 않은 정도를 \hly{팽창계수}로 나타낼 수 있다.
팽창계수가 더 큰 것은 \seolfa{황동}이다.
\ \seolight{알코올 온도계}의 경우 알코올의 열에 의한 팽창으로 온도를 확인할 수 있다.
이때, \seolight{유리는 팽창}하지 않는가? 어떻게 해야 \seolight{정확한 온도계}를 만들 수 있을까?
\end{sssbox}
\memo{두 물질의 팽창을 종합적으로 고려하여 눈금을 작성한다. \\
\seolight{황동(brass)}은 \seolight{구리와 아연}의 합금으로
\seolight{놋쇠}라고 부르기도 한다. \\
합금 정도에 따라 성질이 조금씩 달라질 수도 있지만
구리7:아연3의 \seolight{칠삼황동을 대표}값으로 나타낸다.}%
\clearpage



%%%%%%%%  subsubection  %%%%%%%%%%%%%%%%
\subsubsection{Linear Expansion}
\memo{subsub: 선팽창}%
\ 길이가 $L$인 금속막대의 온도가 $\Delta T$만큼 올라간다면 길이 변화는 다음과 같다.
%% eq box. 선팽창
\begin{eqbox} \Delta L = L \alpha \Delta T
\label{eq:linear_expansion} \end{eqbox}
\ 여기서 $\alpha$는 \hly{\bd{선팽창계수}}라고 한다.
차원을 분석하면 선팽창계수는 \seolfa{$\mathrm{K}^{-1}$}의 단위를 갖는다.
선팽창계수는 \hly{단위온도 변화당 길이 변화의 비율}을 의미한다.
\begin{sssbox}
\memo{이 책에서는 초기 길이를 $L$이라고 함. \\
\seolight{$L_0$으로 생각}하는 것이 나중에 안 헷갈릴수도 있으니 초기 길이를 $L_0$로 쓸 것임. \\
또한 길이는 \seolight{소문자 $l$}로 나타낼 것임. \\
 \\
\seolight{온도 차이}에 곱하므로, 선팽창계수의 단위는 섭씨온도, 절대온도 무엇을 써도 상관 없다. \\
 \\
\seolight{열수축}이라는 게 있을까? \\
그냥 $\Delta L = L \alpha \Delta T$ (\autoref{eq:linear_expansion})에서 \\
\seolight{$\Delta T < 0 $}임. 그대로 감소하는 것.}%
처음 길이를 $l_0$, 나중 길이를 $l$이라고 하고, $l$에 대해 정리하면 다음을 얻는다.
\begin{equation*} l = l_0 + \Delta l = l_0(1 + \alpha T) \end{equation*}
\end{sssbox}

\begin{checkbox*}
겨울에 철로를 매설하며 여름에 늘어날 것을 대비하고자 한다.
철로 한 구간의 길이는 \seolight{$10\,\mathrm{m}$}일 때 얼마의 간격으로 철로를 매설해야 할까?
(단, 겨울의 온도는 \seolight{$-10\,^\circ\mathrm{C}$}이고,
여름의 최대 온도는 \seolight{$40\,^\circ\mathrm{C}$}이다.
철의 선팽창계수는 \seolight{$11 \times 10^{-6}/\mathrm{C}^\circ$}이다.)
\end{checkbox*}

\begin{solbox}
\bnset
\bd{Sol: }
%% Fig. 철로의 열팽창
\seolfig[0.8]{thermal_expansion_of_rails}
\bn $10\,\mathrm{m}$ 막대기를 배치할 때, 틈을 $x$라고 하면, 두 가지 방법으로 생각할 수 있다. \\
\bns \bul \seolight{왼쪽을 모두 고정}시키고 반대쪽(오른쪽)만 늘어난다. \\
\bns \bul 양쪽이 똑같이 늘어서 늘어난 길이를 \seolight{절반 나누지만}, \\
\bns \bls 두 막대가 가까워지므로 \seolight{다시 2를 곱한다}. 즉, \seolight{위와 같다}. \\
\bn $\Delta l = l_0 \alpha \Delta T$에서
$\Delta l = 10 \times 11 \times 10^{-6} \times \seolight{50}
= 5.5 \times 10^{-3}\,(\mathrm{m})$이다. \\
\bd{Ans: } $5.5\,\mathrm{mm}$
\end{solbox}%
\vspace{-10pt}%
\memo{(a) Room temperature values except for the listing for ice. \\
(b) This alloy was designed to have a low coefficient of expansion.
The word is a shortened form of “invariable.” \\
(a) 얼음을 제외하고는 \seolight{실온}에서의 값이다. \\
(b) \seolight{인바 합금}의 팽창계수는 \seolight{낮다}.
인바(Invar)는 영어 \seolight{``invariable''}의 약어이다. \\
 \\
유리도 어떻게 만든 유리인지에 따라 성질이 달라질 것이다. \\
\seolight{일반적으로 사용되는 유리를 대표값}으로 한다.}%
\addlines{1}
%% Tab. 선팽창계수의 예
\somecoefficientsoflinearexpansion
\ \autoref{tab:some_coefficients_of_linear_expansion}\은 여러 물질에서
선팽창계수가 얼마인지를 나타낸 것이다.
\clearpage



%% Fig. 자의 팽창
\thermalexpansionofaruler
\memo{서로 다른 온도에서 강철로 만든 자.
자가 팽창하면 그 위의 척도, 숫자, 자의 두께, 원과 원형 구멍의 지름 등은
\seolight{모두 동일한 비율로 늘어난다}(팽창되었음을 보이기 위해 과장되게 그렸다).}%
\ \autoref{fig:thermal_expansion_of_a_ruler}\은 자의 열팽창을 나타낸 것이다.
\hly{사진을 확대}한 것과 비슷하다. 구멍을 뚫으면서 생긴 원판은 자에 딱 맞고,
자와 원판의 온도가 같이 올라갈 때 원판은 \seolight{여전히 구멍에 딱 맞게} 된다.
\begin{sssbox}
\memo{\seolight{휴대폰으로 확대하듯} 그대로 커진다. \\
안으로 부풀듯이 그린 학생들이 굉장히 많음.
위의 \seolight{자 그림처럼 꽉 찬 금속} 덩어리에 \seolight{매직으로 그림}을 그리고
\seolight{마지막에 잘라낸다고 생각}해도 됨}%
2023학년도 광주과학고등학교 입학전형 2단계 2교시(과학영역) 2번 문제 \\
일부가 끊긴 원 모양의 금속 고리를 가열할 때, 변한 금속 고리의 모양은?
%% Fig. 세 가지 선형 온도척도
\quickfig[0.3]{open-ended_metal_ring}
\end{sssbox}

%%%%%%%%  subsubection  %%%%%%%%%%%%%%%%
\subsubsection{Area Expansion}
\memo{subsub: 면팽창 \\
면팽창은 잘 다루지 않음. 일반물리학 교재에서도 면팽창은 없음.
\seolight{면팽창계수}는 부피팽창계수와 같은 이유로 \seolight{선팽창의 2배}가 됨.}%
\ 물체가 늘어난 면적은 처음 면적과 온도 변화의 곱에 비례한다.

%%%%%%%%  subsubection  %%%%%%%%%%%%%%%%
\subsubsection{Volume Expansion}
\memo{subsub: 부피팽창}%
\ 고체의 경우 온도에 따라 모든 방향으로 팽창하기 때문에 부피가 늘어난다.
액체의 경우는 부피팽창만 의미가 있다.
부피가 $V$인 고체나 액체의 온도가 $\Delta T$만큼 올라갈 때 부피 변화는 다음과 같다.
%% eq box. 부피팽창
\begin{eqbox} \Delta V = V \beta \Delta T
\label{eq:volume_expansion} \end{eqbox}
\memo{선팽창과 마찬가지로 이 식에서는 초기 부피를 $V$이라고 함. \\
\seolight{$V_0$으로 생각}하는 것이 나중에 안 헷갈릴 수도 있음 \\
 \\
선은 $\alpha$, \seolight{부피는 $\beta$}를 씀. \\
면은 건너뛴 것임. $\gamma$ 쓰거나 $\beta$에 첨자를 붙여서 면팽창을 나타내기도 함}%
\ 선팽창과 마찬가지로 팽창계수가 있고, $\beta$는 \hly{부피팽창계수}라고 한다.
부피팽창계수와 선팽창계수 사이에는 다음의 관계가 있다.
%% eq box. 부피팽창과 선팽창의 관계
\begin{eqbox} \beta = 3 \alpha
\label{eq:volume–linear_expansion_relation} \end{eqbox}
\clearpage



\begin{sssbox}
\memo{\seolight{\autoref{eq:linear_expansion}은 초기 길이가 $L$}임에 유의 \\
\seolight{$\Delta L = L_0 \alpha \Delta T$} \\
$\Delta L = L - L_0$ 연립}%
간단하게 $\beta = 3 \alpha$임을 보이자. \\
\bnset
\bn 선팽창 식 \autoref{eq:linear_expansion}에서 나중 길이는
$l =$ \seolfa{$l_0 (1 + \alpha \Delta T)$}이다. \\
\bn 세 변의 길이를 각각 $a + \Delta a$ ,\, $b + \Delta b$ ,\, $c + \Delta c$라고 하면 \\
\bns 부피 변화는 다음과 같이 나타낼 수 있다. \\
\vspace{-22pt}%
\memo{$\Delta a$의 값은 \seolight{매우 작아서}
\seolight{두 번} 곱한 값은 \seolight{무시(근사)} \\
\seolight{테일러 급수전개로 근사}하는 것과 같음.}%
\begin{align*}
     \Delta V &= \seolfb{$(a + \Delta a)(b + \Delta b)(c + \Delta c) - abc$} \\
     &\approx \seolfb{$ab \Delta c + bc \Delta a + ca \Delta b$} \\
     &= \seolfb{$3abc \,\alpha \Delta T = V_0 \cdot 3\alpha \cdot \Delta T
     = V_0 \cdot \beta \cdot \Delta T$}
\end{align*}
\vspace{-22pt}%

\bn 비등방성 재료(결정 구조 방향에 따라 다른 팽창률)에서는 \\
\bns $\Delta V = V_0 (\alpha_x + \alpha_y + \alpha_z) \Delta T$가 된다.
\end{sssbox}

\begin{checkbox}
The figure here shows four rectangular metal plates,
with sides of \seolight{$L$}, \seolight{$2L$}, or \seolight{$3L$}.
They are all made of the same material,
and their temperature is to be increased by the same amount. \\
Rank the plates according to the expected increase in \\
(a) their \seolight{vertical heights} and \\
(b) their \seolight{areas}, greatest first. \\
\memo{오른편 그림처럼 변의 길이가 각각 \seolight{$L$}, \seolight{$2L$},
\seolight{$3L$}인 직사각형 금속판이 있다.
각 판은 모두 동일한 재질로 들었고, 온도가 동일하게 증가한다고 하자. \\
(a) 수직 방향 \seolight{높이의 증가}와 \\
(b) \seolight{면적의 증가}가 가장 큰 순서대로 각각 나열하여라.}%

%% Fig. 네 개의 강철 판
\quickfig{four_rectangular_metal_plates}
\end{checkbox}

\begin{solbox}
\bnset
\bd{Sol: } \\
\bn 높이를 보면 $2L$, $3L$, $3L$, $L$인 듯함 \\
\bns 처음 크기에 비례하여 커지므로, 높이 증가는 2 = 3 > 1 > 4 \\
\bn 면적은 $2L^2$, $3L^2$, $6L^2$, $2L^2$인 듯함 \\
\bns 처음 크기에 비례하여 커지므로, 면적 증가는 3 > 2 > 1 = 4 \\
\bd{Ans(a): } 2 = 3 > 1 > 4 \\
\bd{Ans(b): } 3 > 2 > 1 = 4
\end{solbox}

\ 액체는 그릇에 담는다. 따라서 액체가 팽창할 때는 그릇(고체)의 팽창을 함께 고려해야 한다.
우리가 눈으로 관찰하는 팽창은 액체만의 실제 팽창에서 그릇의 팽창을 뺀 \hly{겉보기 팽창}이다.

\begin{sssbox}
\hly{열팽창과 밀도}의 관계는 어떻게 나타낼 수 있을까? \\
온도가 상승해도 질량 $m = \rho_0 V_0 = \rho V $는 변하지 않는다.
\vspace{-8pt}%
\begin{align*}
    V &= V_0 (1 + \beta \Delta T) \\
    \frac{m}{\rho} &= \frac{m}{\rho_0} (1 + \beta \Delta T) \\
    \rho &= \rho_0 \frac{1}{1 + \beta \Delta T} ~ \approx ~ \rho_0(1 - \beta \Delta T)
\end{align*}
\memo{\seolight{$\beta \ll 1$에서 테일러 급수전개} 하였음 \\
\seolight{밀도}는 온도가 증가하면 \seolight{반대로 작아진다}.(팽창하니까)}%
따라서 밀도 변화는 $\Delta \rho = \rho - \rho_0 =$ \seolight{$-$} $\rho_0 \beta \Delta T$이다.
\end{sssbox}
\clearpage



\begin{practicebox}{Thermal expansion on the Moon}
%% Fig. 아폴로15
\apollofifteen
When Apollo 15 landed on the Moon at the foot of
the Apennines mountain range, an American flag was
planted (\autoref{fig:apollo_fifteen}).
The aluminum, telescoping flagpole
was \seolight{$2.0\,\mathrm{m}$} long with a coefficient of linear expansion
\seolight{$2.3 \times 10^{-5}\,/^\circ\mathrm{C}$}. At that latitude on the Moon
($26.1\,^\circ\mathrm{N}$), the temperature varied from \seolight{$290\,\mathrm{K}$}
in the day to \seolight{$110\,\mathrm{K}$} in the night.
What was the \seolight{change in length} of the pole between day and night? \\\vspace{-12pt}%
\memo{달 위의 열팽창 \\
 \\
Apollo 15호가 달의 Apennines 산맥 기슭에 도착하여 미국 국기를 꽂았다
(\autoref{fig:apollo_fifteen}). 망원경이 장착되어 있는 알루미늄 깃대의
길이는 \seolight{$2.0\,\mathrm{m}$}이고, 
선팽창률은 \seolight{$2.3 \times 10^{-5}\,/^\circ\mathrm{C}$}이다.
이 위도($26.1\,^\circ\mathrm{N}$)의 달에서 온도는 낮에는
\seolight{$290\,\mathrm{K}$}, 밤에는 \seolight{$110\,\mathrm{K}$}로 변한다.
낮과 밤 사이에 깃대의 \seolight{길이 변화}는 얼마인가?}%
\end{practicebox}

\begin{solbox}
\bnset
\bd{Sol: } \\
$L_0 = 2.0\,\mathrm{m}$ ,\,
$\alpha = 2.3 \times 10^{-5}\,/\mathrm{K}$ ,\,
$\Delta T = 180\,\mathrm{K}$이므로, \\
\solight{$\Delta L = L_0 \alpha \Delta T$}에서,
$8.28 \times 10^{-3}\,\mathrm{m}$이다. \\
\bd{Ans: } $8.28 \times 10^{-3}\,\mathrm{m}$
\end{solbox}

\begin{practicebox}{The shrinking fuel load}
On a hot day in Las Vegas, a fuel trucker loaded \seolight{$\num{37000}\,\mathrm{L}$}
of diesel fuel. He encountered cold weather on the way to
Payson, Utah, where the temperature was \seolight{$23.0\,\mathrm{K}$} lower
than in Las Vegas, and where he delivered his entire load.
How many liters did he deliver? The coefficient of volume expansion
for diesel fuel is \seolight{$\beta = 9.50 \times 10^{-4}\,/\mathrm{C}^\circ$},
and the coefficient of linear expansion for his steel tank is
\seolight{$\alpha = 11 \times 10^{-6}\,/\mathrm{C}^\circ$}. \\\vspace{-12pt}%
\memo{부피 열팽창 \\
 \\
무더운 날 유류 수송차량 운전기사가 라스베이거스에서
\seolight{$\num{37000}\,\mathrm{L}$}의 경유를 싣고 유타 주의 페이손으로 배달을 갔다.
페이손에서는 날씨가 쌀쌀해서 라스베이거스보다 온도가 \seolight{$23.0\,\mathrm{K}$}
낮았다면, 배달된 경유는 몇 리터인가?
경유의 부피 팽창계수는 \seolight{$9.50 \times 10^{-4}\,/\mathrm{C}^\circ$}이고,
트럭의 강철 탱크의 선팽창 계수는
\seolight{$11 \times 10^{-6}\,/\mathrm{C}^\circ$}라고 하자. \\
 \\
겉보기 팽창 문제가 아니다.}%
\end{practicebox}

\begin{solbox}
\bnset
\bd{Sol: } \\
\bn 온도가 낮아져서 경유의 부피가 감소한다. \\
\bns 이때, \solight{강철 탱크 역시 부피가 감소}한다. \\
\bns 그러나 경유의 배달과 강철 탱크의 부피는 \solight{상관 없다}.
(경유 부피 변화만 고려) \\
\bn 경유의 변화량은 \solight{$\Delta V = V_0 \beta \Delta T$}에서 \\
\bns $\Delta V = (\num{37000}\,\mathrm{L})(9.50 \times 10^{-4}/\mathrm{K})(-23.0\,\mathrm{K}) = -808.45\,\mathrm{L}$ \\
\bn 배달된 양은 $\num{37000} - 808.45 = \num{36191.5}\,\mathrm{L}$이다. \\
\bd{Ans: } $3.62 \times 10^4 \,\mathrm{L}$
\end{solbox}
\clearpage



\begin{practicebox*}{자의 선팽창}
\autoref{tab:some_coefficients_of_linear_expansion}\을 참고하면,
\seolight{강철(Steel)}의 선팽창계수는 \seolight{$11 \times 10^{-6}\,/\mathrm{C}^\circ$},
\seolight{황동(Brass)}의 선팽창계수는 \seolight{$19 \times 10^{-6}\,/\mathrm{C}^\circ$}이다.
\seolight{$0\,^\circ\mathrm{C}$}에서 정확히 눈금을 맞춘 강철 자로
\seolight{$30\,^\circ\mathrm{C}$}일 때 황동 막대의 길이를
\seolight{$100\,\mathrm{cm}$}로 측정하였다.
\seolight{$0\,^\circ\mathrm{C}$}에서 황동 막대의 길이는 얼마인가?
\end{practicebox*}
\vspace{3cm}%

\begin{practicebox*}{유리그릇 속 액체}
\autoref{tab:some_coefficients_of_linear_expansion}\을 참고하면,
\seolight{유리(Glass)}의 선팽창계수는 \seolight{$9 \times 10^{-6}\,/\mathrm{C}^\circ$}이다.
부피팽창계수가 \seolight{$2 \times 10^{-4}\,/\mathrm{C}^\circ$}로 유지되는 액체가
유리그릇에 담겨있다. \\
\memo{물만 봐도 {$20\,^\circ\mathrm{C}$},
{$100\,^\circ\mathrm{C}$}일때 팽창계수 몇 배로 달라짐 \\
약 \seolight{$2.07 \times 10^{-4}\,/\mathrm{C}^\circ$} \\
약 \seolight{$6.9 \times 10^{-4}\,/\mathrm{C}^\circ$} \\
계속 일정하게 유지된다고 가정하자.}%
\anset
\an \seolight{$20\,^\circ\mathrm{C}$}에서 부피가
\seolight{$1\,\mathrm{m^3}$}인 정육면체 모양의 유리그릇에 액체가 가득 들어 있다. \\
\ans 액체와 그릇의 온도를 함께 \seolight{$100\,^\circ\mathrm{C}$}로 올리면 각각 부피가 얼마가 되는가? \\
\an \seolight{$100\,^\circ\mathrm{C}$}에서 부피가
\seolight{$1\,\mathrm{m^3}$}인 정육면체 모양의 유리그릇에 액체가 가득 들어 있다. \\
\ans 액체와 그릇의 온도를 함께 \seolight{$20\,^\circ\mathrm{C}$}로 내리면 각각 부피가 얼마가 되는가?
\end{practicebox*}
\vspace{3cm}%

\begin{practicebox*}{팽창 보정 장치}
온도가 변하더라도 떨어진 두 점의 거리가 일정하게 하는 방법은 선팽창계수가 다른 두 막대의 한쪽 끝을
결합하는 것이다. 금속 \seolight{A}의 질량은 $m_\text{A}$, 선팽창계수는 $\alpha_\text{A}$이고,
금속 \seolight{B}의 질량은 $m_\text{B}$, 선팽창계수는 $\alpha_\text{B}$이다.
%% Fig. 팽창 보정 장치
\quickfig{thermal_compensation_device}
\anset
\an 온도변화에 따라 길이 $L$이 변하지 않기 위한 조건을 구하시오. \\
\an 물질 \seolight{A는 황동}, 물질 \seolight{B는 강철}이다.
\seolight{$0\,^\circ\mathrm{C}$}에서 황동의 길이가
\seolight{$250\,\mathrm{cm}$}이면 \\
\ans $L$은 얼마인가?
\end{practicebox*}
\clearpage



\insertTeacherPages
{ % 교사용 페이지 1 시작
\begin{solbox}
\bnset
\bd{Sol: } \\
\bd{Ans: } 
\end{solbox}

\begin{solbox}
\bnset
\bd{Sol: } \\
\bd{Ans: } 
\end{solbox}
} % 교사용 페이지 1 끝
{ % 교사용 페이지 2 시작
\begin{solbox}
\bnset
\bd{Sol: } \\
\bd{Ans: } 
\end{solbox}
} % 교사용 페이지 2 끝
\clearpage