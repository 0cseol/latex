%%%%  section  %%%%%%%%%%%%%%%%%%%%%%%%%
\section{Avogadro's Number}
\memo{chapter: 기체운동론 \\ section: Avogadro 수}%
\ 본문 내용

%%%%%%  subsection  %%%%%%%%%%%%%%%%%%%%
\subsection{Avogadro's Number}
\memo{sub: Avogadro 수}%
\ 본문 내용

\begin{equation} N_A = 6.02 \times 10^{23} \ \text{mol}^{-1} ~~~~~ \text{(Avogadro's number)} \end{equation}

%% eq box. 몰수
\begin{eqbox} n = \frac{N}{N_A}
\label{eq:number_of_moles} \end{eqbox}

%% eq box. 몰수와 질량의 관계
\begin{eqbox} n = \frac{M_{\text{sam}}}{M} = \frac{M_{\text{sam}}}{m N_A}
\label{eq:relationship_between_number_of_moles_and_mass} \end{eqbox}

\begin{equation} M = m N_A \end{equation}

%%%%  section  %%%%%%%%%%%%%%%%%%%%%%%%%
\section{Ideal Gases}
\memo{section: 이상기체}%
\ 본문 내용

%%%%%%  subsection  %%%%%%%%%%%%%%%%%%%%
\subsection{Ideal Gases}
\memo{sub: 이상기체}%
\ 본문 내용

%% eq box. 이상기체 법칙
\begin{eqbox} pV = nR\,T ~~~~~ \text{(ideal gas law)}
\label{eq:ideal_gas_law} \end{eqbox}

\begin{equation} R = 8.31\ \mathrm{J/mol \cdot K} \end{equation}

\begin{equation} k = \frac{R}{N_A} = \frac{8.31\ \mathrm{J/mol \cdot K}}{6.02 \times 10^{23}\ \mathrm{mol^{-1}}} = 1.38 \times 10^{-23}\ \mathrm{J/K} \end{equation}

\begin{equation} nR = Nk \end{equation}

%% eq box. 이상기체 법칙 두 번째 표현
\begin{eqbox} pV = Nk\,T ~~~~~ \text{(ideal gas law)}
\label{eq:second_expression_for_ideal_gas_law} \end{eqbox}

%%%%%%%%  subsubection  %%%%%%%%%%%%%%%%
\subsubsection{Work Done by an Ideal Gas at Constant Temperature}
\memo{subsub: 온도가 일정할 때 이상기체가 한 일}%
\ 본문 내용

\begin{equation} p = nR\,T \frac{1}{V} = (\text{a constant}) \frac{1}{V} \end{equation}

\begin{equation} W = \int_{V_i}^{V_f} p\, dV \end{equation}

\begin{equation} W = \int_{V_i}^{V_f} \frac{nR\,T}{V}\, dV \end{equation}

\begin{equation} W = nR\,T \int_{V_i}^{V_f} \frac{dV}{V} = W = nR\,T \Bigg[ \ln V \Bigg]_{V_i}^{V_f}
 \end{equation}

%% eq box. 이상기체가 등온팽창 동안에 한 일
\begin{eqbox} W = nR\,T \ln \frac{V_f}{V_i} ~~~~~ \text{(ideal gas, isothermal process)}
\label{eq:work_done_by_an_ideal_gas_during_an_isothermal_expansion} \end{eqbox}

%%%%%%%%  subsubection  %%%%%%%%%%%%%%%%
\subsubsection{Work Done at Constant Volume and at Constant Pressure}
\memo{subsub: 압력과 부피가 일정할 때 한 일 \\ 이상기체가 한 일임. 누가 한 일인지 써줘야 함}%
\ 본문 내용

\begin{equation} W = 0 \quad \text{(constant-volume process)} \end{equation}

\begin{equation} W = p(V_f - V_i) = p\, \Delta V ~~~~~ \text{(constant-pressure process)} \end{equation}

%%%%  section  %%%%%%%%%%%%%%%%%%%%%%%%%
\section{Pressure, Temperature, and RMS Speed}
\memo{section: 압력, 온도 및 제곱평균제곱근 속력}%
\ 본문 내용

%%%%%%  subsection  %%%%%%%%%%%%%%%%%%%%
\subsection{Pressure, Temperature, and RMS Speed}
\memo{sub: 압력, 온도 및 제곱평균제곱근 속력}%
\ 본문 내용

\begin{equation*} \Delta p_x = (-mv_x) - (mv_x) = -2mv_x \end{equation*}

\begin{equation*} \frac{\Delta p_x}{\Delta t} = \frac{2mv_x}{2L/v_x} = \frac{mv_x^2}{L} \end{equation*}

\begin{align}
    p&= \frac{F_x}{L^2} = \frac{mv_{x1}^2 / L + mv_{x2}^2 / L + \cdots + mv_{xN}^2 / L}{L^2} \notag \\
    &= \left( \frac{m}{L^3} \right)(v_{x1}^2 + v_{x2}^2 + \cdots + v_{xN}^2)
\end{align}

\begin{equation*} p = \frac{nmN_A}{L^3} \left( v_x^2 \right)_{\text{avg}} \end{equation*}

\begin{equation} p = \frac{nM (v_x^2)_{\text{avg}}}{V} \end{equation}

\begin{equation} p = \frac{nM (v^2)_{\text{avg}}}{3V} \end{equation}

\begin{equation} p = \frac{nM v_{\text{rms}}^2}{3V} \end{equation}

%% eq box. rms속력
\begin{eqbox} v_{\text{rms}} = \sqrt{\frac{3R\,T}{M}}
\label{eq:root-mean-square_speed} \end{eqbox}

%%%%  section  %%%%%%%%%%%%%%%%%%%%%%%%%
\section{Translational Kinetic Energy}
\memo{section: 병진 운동에너지}%
\ 본문 내용

%%%%%%  subsection  %%%%%%%%%%%%%%%%%%%%
\subsection{Translational Kinetic Energy}
\memo{sub: 병진 운동에너지}%
\ 본문 내용

\begin{equation} K_{\text{avg}} = \left( \tfrac{1}{2}mv^2 \right)_{\text{avg}} = \tfrac{1}{2}m (v^2)_{\text{avg}} = \tfrac{1}{2}m v_{\text{rms}}^2 \end{equation}

\begin{equation*} K_{\text{avg}} = \left( \tfrac{1}{2}m \right) \frac{3R\,T}{M} \end{equation*}

\begin{equation*} K_{\text{avg}} = \frac{3R\,T}{2N_A} \end{equation*}

%% eq box. 평균 병진 운동에너지
\begin{eqbox} K_{\text{avg}} = \tfrac{3}{2}k\,T
\label{eq:average_translational_kinetic_energy} \end{eqbox}

%%%%  section  %%%%%%%%%%%%%%%%%%%%%%%%%
\section{Mean Free Path}
\memo{section: 평균자유거리}%
\ 본문 내용

%%%%%%  subsection  %%%%%%%%%%%%%%%%%%%%
\subsection{Mean Free Path}
\memo{sub: 평균자유거리}%
\ 본문 내용

%% eq box. 평균 자유 거리
\begin{eqbox} \lambda = \frac{1}{\sqrt{2}\, \pi d^2 \, N/V} ~~~~~ \text{(mean free path)}
\label{eq:mean_free_path} \end{eqbox}

\begin{align}
    \lambda&= \frac{\text{length of path during } \Delta t}{\text{number of collisions in } \Delta t} ~ \approx ~ \frac{v \, \Delta t}{\pi d^2 v \, \Delta t \, N/V} \notag \\
    &= \frac{1}{\pi d^2 \, N/V}
\end{align}

%%%%  section  %%%%%%%%%%%%%%%%%%%%%%%%%
\section{The Distribution of Molecular Speeds}
\memo{section: 분자의 속력 분포}%
\ 본문 내용

%%%%%%  subsection  %%%%%%%%%%%%%%%%%%%%
\subsection{The Distribution of Molecular Speeds}
\memo{sub: 분자의 속력 분포}%
\ 본문 내용

%% eq box. Maxwell 속력분포 법칙
\begin{eqbox} P(v) = 4\pi \left( \frac{M}{2\pi R\,T} \right)^{3/2} v^2 e^{-Mv^2 / 2RT}
\label{eq:Maxwell's_speed_distribution_law} \end{eqbox}

\begin{equation} \int_0^{\infty} P(v) \, dv = 1 \end{equation}

\begin{equation} \text{frac} = \int_{v_1}^{v_2} P(v) \, dv \end{equation}

%%%%%%%%  subsubection  %%%%%%%%%%%%%%%%
\subsubsection{Average, RMS, and Most Probable Speeds}
\memo{subsub: 평균속력, RMS 속력, 가장 잦은 속력 \\ 최빈속력이라고 보통 부름}%
\ 본문 내용

\begin{equation} v_{\text{avg}} = \int_0^{\infty} v\, P(v)\, dv \end{equation}

%% eq box. 기체 분자의 평균속력
\begin{eqbox} v_{\text{avg}} = \sqrt{\frac{8R\,T}{\pi M}} ~~~~~ \text{(average speed)}
\label{eq:average_speed_of_the_molecules_in_a_gas} \end{eqbox}

\begin{equation} (v^2)_{\text{avg}} = \int_0^{\infty} v^2\, P(v)\, dv \end{equation}

\begin{equation} (v^2)_{\text{avg}} = \frac{3R\,T}{M} \end{equation}

%% eq box. rms 속력
\begin{eqbox} v_{\text{rms}} = \sqrt{(v^2)_{\text{avg}}} = \sqrt{\frac{3R\,T}{M}} \quad \text{(rms speed)}
\label{eq:rms_speed} \end{eqbox}

%% eq box. 가장 잦은 속력
\begin{eqbox} v_p = \sqrt{\frac{2R\,T}{M}} ~~~~~ \text{(most probable speed)}
\label{eq:most_probable_speed} \end{eqbox}

%%%%  section  %%%%%%%%%%%%%%%%%%%%%%%%%
\section{The Molar Specific Heats of an Ideal Gas}
\memo{section: 이상기체의 몰비열}%
\ 본문 내용

%%%%%%  subsection  %%%%%%%%%%%%%%%%%%%%
\subsection{The Molar Specific Heats of an Ideal Gas}
\memo{sub: 이상기체의 몰비열}%
\ 본문 내용

%%%%%%%%  subsubection  %%%%%%%%%%%%%%%%
\subsubsection{Internal Energy \( E_{\text{int}} \)}
\memo{subsub: 내부에너지 \( E_{\text{int}} \)}%
\ 본문 내용

\begin{equation} E_{\text{int}} = (n N_A) K_{\text{avg}} = (n N_A) \left( \tfrac{3}{2} k\,T \right) \end{equation}

%% eq box. 단원자 이상기체의 내부에너지
\begin{eqbox} E_{\text{int}} = \tfrac{3}{2} nR\,T ~~~~~ \text{(monatomic ideal gas)}
\label{eq:internal_energy_of_monatomic_ideal_gas} \end{eqbox}

%%%%%%%%  subsubection  %%%%%%%%%%%%%%%%
\subsubsection{Molar Specific Heat at Constant Volume}
\memo{subsub: 부피가 일정할 때의 몰비열}%
\ 본문 내용

%% eq box. 등적 과정에서 열과 온도변화 관계
\begin{eqbox} Q = n C_V \, \Delta T ~~~~~ \text{(constant volume)}
\label{eq:heat_and_temperature_change_at_constant_volume} \end{eqbox}

\begin{equation} \Delta E_{\text{int}} = n C_V \, \Delta T - W \end{equation}

\begin{equation} C_V = \frac{\Delta E_{\text{int}}}{n \, \Delta T} \end{equation}

\begin{equation} \Delta E_{\text{int}} = \tfrac{3}{2} nR\, \Delta T \end{equation}

%% eq box. 단원자 이상기체의 등적 몰비열
\begin{eqbox} C_V = \tfrac{3}{2} R = 12.5\ \mathrm{J/mol \cdot K} ~~~~~ \text{(monatomic gas)}
\label{eq:molar_specific_heat_at_constant_volume} \end{eqbox}

\begin{equation} E_{\text{int}} = n C_V T ~~~~~ \text{(any ideal gas)} \end{equation}

%% eq box. 이상기체의 내부에너지 변화
\begin{eqbox} \Delta E_{\text{int}} = n C_V \, \Delta T ~~~~~ \text{(ideal gas, any process)}
\label{eq:change_in_the_internal_energy_Eint_of_a_confined_ideal_gas} \end{eqbox}

%%%%%%%%  subsubection  %%%%%%%%%%%%%%%%
\subsubsection{Molar Specific Heat at Constant Pressure}
\memo{subsub: 압력이 일정할 때의 몰비열}%
\ 본문 내용

\begin{equation} Q = n C_p \, \Delta T ~~~~~ \text{(constant pressure)} \end{equation}

\begin{equation} \Delta E_{\text{int}} = Q - W \end{equation}

%% eq box. 기체가 한 일과 이상기체 방정식
\begin{eqbox} W = p \, \Delta V = n R \, \Delta T
\label{eq:work_done_by_a_gas_and_the_ideal_gas_law} \end{eqbox}

\begin{equation*} C_V = C_p - R \end{equation*}

%% eq box. 등압 몰비열과 등적 몰비열의 관계
\begin{eqbox} C_p = C_V + R
\label{eq:molar_specific_heats_at_constant_pressure_and_constant_volume} \end{eqbox}

%%%%  section  %%%%%%%%%%%%%%%%%%%%%%%%%
\section{Degrees of Freedom and Molar Specific Heats}
\memo{section: 자유도와 몰비열}%
\ 본문 내용

%%%%%%  subsection  %%%%%%%%%%%%%%%%%%%%
\subsection{Degrees of Freedom and Molar Specific Heats}
\memo{sub: 자유도와 몰비열}%
\ 본문 내용

\begin{equation} C_V = \left( \frac{f}{2} \right) R = 4.16f\ \mathrm{J/mol \cdot K} \end{equation}

%%%%%%  subsection  %%%%%%%%%%%%%%%%%%%%
\subsection{A Hint of Quantum Theory}
\memo{sub: 양자론의 단서}%
\ 본문 내용

%%%%  section  %%%%%%%%%%%%%%%%%%%%%%%%%
\section{The Adiabatic Expansion of an Ideal Gas}
\memo{section: 이상기체의 단열팽창}%
\ 본문 내용

%%%%%%  subsection  %%%%%%%%%%%%%%%%%%%%
\subsection{The Adiabatic Expansion of an Ideal Gas}
\memo{sub: 이상기체의 단열팽창}%
\ 본문 내용

%% eq box. 단열과정에서 압력과 부피 사이의 관계
\begin{eqbox} pV^{\gamma} = \text{a constant} ~~~~~ \text{(adiabatic process)}
\label{eq:pressure_and_volume_in_an_adiabatic_process} \end{eqbox}

단열과정에서 압력과 부피 사이의 관계식(\autoref{eq:pressure_and_volume_in_an_adiabatic_process})은


\begin{equation} p_i V_i^\gamma = p_f V_f^\gamma ~~~~~ \text{(adiabatic process)} \end{equation}

\begin{equation*} \left( \frac{nR\,T}{V} \right) V^\gamma = \text{a constant} \end{equation*}

%% eq box. 단열과정에서 온도와 부피 사이의 관계
\begin{eqbox} T V^{\gamma - 1} = \text{a constant} ~~~~~ \text{(adiabatic process)}
\label{eq:temperature_and_volume_in_an_adiabatic_process} \end{eqbox}

\begin{equation} T_i V_i^{\gamma - 1} = T_f V_f^{\gamma - 1} ~~~~~ \text{(adiabatic process)} \end{equation}

%%%%%%%%  subsubection  %%%%%%%%%%%%%%%%
\subsubsection{Proof of \autoref{eq:pressure_and_volume_in_an_adiabatic_process}}
\memo{subsub: \autoref{eq:pressure_and_volume_in_an_adiabatic_process}의 증명}%
\ 본문 내용


\begin{equation} dE_{\text{int}} = Q - p\,dV \end{equation}

\begin{equation} n\,dT = -\left( \frac{p}{C_V} \right) dV \end{equation}

\begin{equation} p\,dV + V\,dp = nR\,dT \end{equation}

\begin{equation} n\,dT = \frac{p\,dV + V\,dp}{C_p - C_V} \end{equation}

\begin{equation*} \frac{dp}{p} + \left( \frac{C_p}{C_V} \right) \frac{dV}{V} = 0 \end{equation*}

\begin{equation*} \ln p + \gamma \ln V = \text{a constant} \end{equation*}

\begin{equation} pV^{\gamma} = \text{a constant} \end{equation}

%%%%%%%%  subsubection  %%%%%%%%%%%%%%%%
\subsubsection{Free Expansions}
\memo{subsub: 자유팽창}%
\ 본문 내용

\begin{equation} T_i = T_f ~~~~~ \text{(free expansion)} \end{equation}

\begin{equation} p_i V_i = p_f V_f ~~~~~ \text{(free expansion)} \end{equation}