%%%%  section  %%%%%%%%%%%%%%%%%%%%%%%%%
\section{Entropy}
\memo{chapter: 엔트로피와 열역학 제2법칙 \\ section: 엔트로피}%
\ 본문 내용

%%%%%%  subsection  %%%%%%%%%%%%%%%%%%%%
\subsection{Irreversible Processes and Entropy}
\memo{sub: 비가역과정과 엔트로피}%
\ 본문 내용

%%%%%%  subsection  %%%%%%%%%%%%%%%%%%%%
\subsection{Change in Entropy}
\memo{sub: 엔트로피 변화}%
\ 본문 내용

%% eq box. 엔트로피 변화의 정의
\begin{eqbox} \Delta S = S_f - S_i = \int_i^f \frac{dQ}{T} ~~~~~ \text{(change in entropy defined)}
\label{eq:change_in_entropy_defined} \end{eqbox}

\begin{equation*} \Delta S = S_f - S_i = \frac{1}{T} \int_i^f dQ \end{equation*}

\begin{equation} \Delta S = S_f - S_i = \frac{Q}{T} ~~~~~ \text{(change in entropy, isothermal process)} \end{equation}

\begin{equation} \Delta S = S_f - S_i ~ \approx ~ \frac{Q}{T_{\text{avg}}} \end{equation}

%%%%%%%%  subsubection  %%%%%%%%%%%%%%%%
\subsubsection{Entropy as a State Function}
\memo{subsub: 상태함수로서 엔트로피}%
\ 본문 내용

\begin{equation*} dE_\mathrm{int} = dQ - dW \end{equation*}

\begin{equation*} dQ = p\,dV + nC_V\,dT \end{equation*}

\begin{equation*} \frac{dQ}{T} = nR\,\frac{dV}{V} + nC_V\,\frac{dT}{T} \end{equation*}

\begin{equation*} \int_i^f \frac{dQ}{T} = \int_i^f nR\,\frac{dV}{V} + \int_i^f nC_V\,\frac{dT}{T} \end{equation*}

%% eq box. 이상기체의 엔트로피 변화
\begin{eqbox} \Delta S = S_f - S_i = nR \ln\frac{V_f}{V_i} + nC_V \ln\frac{T_f}{T_i}
\label{eq:change_in_entropy_of_an_ideal_gas} \end{eqbox}

%%%%%%  subsection  %%%%%%%%%%%%%%%%%%%%
\subsection{The Second Law of Thermodynamics}
\memo{sub: 열역학 제2법칙}%
\ 본문 내용

\begin{equation*} \Delta S_{\text{gas}} = -\frac{|Q|}{T} \end{equation*}

\begin{equation*} \Delta S_{\text{res}} = +\frac{|Q|}{T} \end{equation*}

%% eq box. 열역학 제2법칙
\begin{eqbox} \Delta S \geq 0 ~~~~~ \text{(second law of thermodynamics)}
\label{eq:second_law_of_thermodynamics} \end{eqbox}

%%%%%%%%  subsubection  %%%%%%%%%%%%%%%%
\subsubsection{Force Due to Entropy}
\memo{subsub: 엔트로피 때문에 생기는 힘}%
\ 본문 내용

\begin{equation*} dE = dQ - dW \end{equation*}

\begin{equation} dE = T\, dS + F\, dx \end{equation}

\begin{equation} F = -T \frac{dS}{dx} \end{equation}

%%%%  section  %%%%%%%%%%%%%%%%%%%%%%%%%
\section{Entropy in the Real World: Engines}
\memo{section: 일상생활의 엔트로피: 기관}%
\ 본문 내용

%%%%%%  subsection  %%%%%%%%%%%%%%%%%%%%
\subsection{Entropy in the Real World: Engines}
\memo{sub: 일상생활의 엔트로피: 기관}%
\ 본문 내용

%%%%%%%%  subsubection  %%%%%%%%%%%%%%%%
\subsubsection{A Carnot Engine}
\memo{subsub: Carnot 기관 \\ 카르노기관}%
\ 본문 내용

\begin{equation} W = \left| Q_H \right| - \left| Q_L \right| \end{equation}

\begin{equation} \Delta S = \Delta S_H + \Delta S_L = \frac{ \left| Q_H \right| }{ T_H } - \frac{ \left| Q_L \right| }{ T_L } \end{equation}

\begin{equation} \frac{ \left| Q_H \right| }{ T_H } = \frac{ \left| Q_L \right| }{ T_L } \end{equation}

%%%%%%%%  subsubection  %%%%%%%%%%%%%%%%
\subsubsection{Efficiency of a Carnot Engine}
\memo{subsub: Carnot 기관의 효율}%
\ 본문 내용

%% eq box. 기관의 효율
\begin{eqbox} \varepsilon = \frac{\text{energy we get}}{\text{energy we pay for}} = \frac{\left| W \right|}{\left| Q_H \right|} ~~~~~ \text{(efficiency, any engine)}
\label{eq:efficiency_of_any_engine} \end{eqbox}

\begin{equation} \varepsilon = \frac{\left| Q_H \right| - \left| Q_L \right|}{\left| Q_H \right|} = 1 - \frac{\left| Q_L \right|}{\left| Q_H \right|} \end{equation}

%% eq box. 카르노기관의 효율
\begin{eqbox} \varepsilon_C = 1 - \frac{T_L}{T_H} ~~~~~ \text{(efficiency, Carnot engine)}
\label{eq:efficiency_of_carnot_engine} \end{eqbox}

%%%%%%%%  subsubection  %%%%%%%%%%%%%%%%
\subsubsection{Stirling Engine}
\memo{subsub: Stirling 기관 \\ 스털링엔진, 스털링기관}%
\ 본문 내용

%%%%  section  %%%%%%%%%%%%%%%%%%%%%%%%%
\section{Refrigerators and Real Engines}
\memo{section: 냉동기와 실제 기관}%
\ 본문 내용

%%%%%%  subsection  %%%%%%%%%%%%%%%%%%%%
\subsection{Entropy in the Real World: Refrigerators}
\memo{sub: 일상생활의 엔트로피: 냉동기}%
\ 본문 내용

%% eq box. 일반 냉장고 성능 계수
\begin{eqbox} K = \frac{\text{what we want}}{\text{what we pay for}} = \frac{\left| Q_L \right|}{\left| W \right|} ~~~~~ \text{\parbox{4.4cm}{(coefficient of performance, \\\hspace*{1pt} any refrigerator)}}
\label{eq:coefficient_of_performance_of_any_refrigerator} \end{eqbox}

\begin{equation} K = \frac{\left| Q_L \right|}{\left| Q_H \right| - \left| Q_L \right|} \end{equation}

%% eq box. 카르노 냉장고 성능 계수
\begin{eqbox} K_C = \frac{T_L}{T_H - T_L} ~~~~~ \text{\parbox{4.4cm}{(coefficient of performance, \\\hspace*{1pt} Carnot refrigerator)}}
\label{eq:coefficient_of_performance_of_carnot_refrigerator} \end{eqbox}

\begin{equation*} \Delta S = -\frac{\left| Q \right|}{T_L} + \frac{\left| Q \right|}{T_H} \end{equation*}

%%%%%%  subsection  %%%%%%%%%%%%%%%%%%%%
\subsection{The Efficiencies of Real Engines}
\memo{sub: 실제 기관의 효율}%
\ 본문 내용

\begin{equation} \varepsilon_X > \varepsilon_C ~~~~~ \text{(a claim)} \end{equation}

\begin{equation*} \frac{ \left| W \right| }{ \left| Q'_{\mathrm{H}} \right| } > \frac{ \left| W \right| }{ \left| Q_{\mathrm{H}} \right| } \end{equation*}

\begin{equation} \left| Q_{\mathrm{H}} \right| > \left| Q'_{\mathrm{H}} \right| \end{equation}

\begin{equation*} \left| Q_{\mathrm{H}} \right| - \left| Q_{\mathrm{L}} \right| = \left| Q'_{\mathrm{H}} \right| - \left| Q_{\mathrm{L}} \right| \end{equation*}

\begin{equation} \left| Q_{\mathrm{H}} \right| - \left| Q'_{\mathrm{H}} \right| = \left| Q_{\mathrm{L}} \right| - \left| Q'_{\mathrm{L}} \right| = Q \end{equation}

%%%%  section  %%%%%%%%%%%%%%%%%%%%%%%%%
\section{A Statistical View of Entropy}
\memo{section: 통계역학적 관점에서 본 엔트로피}%
\ 본문 내용

%%%%%%  subsection  %%%%%%%%%%%%%%%%%%%%
\subsection{A Statistical View of Entropy}
\memo{sub: 통계역학적 관점에서 본 엔트로피}%
\ 본문 내용

\begin{equation*} W_{\text{III}} = \frac{6!}{4! \, 2!} = \frac{720}{24 \times 2} = 15 \end{equation*}

%% eq box. 배열에 대한 경우의 수
\begin{eqbox} W = \frac{N!}{n_1! \, n_2!} ~~~~~ \text{(multiplicity of configuration)}
\label{eq:multiplicity_of_configuration} \end{eqbox}

%%%%%%%%  subsubection  %%%%%%%%%%%%%%%%
\subsubsection{Probability and Entropy}
\memo{subsub: 확률과 엔트로피}%
\ 본문 내용

%% eq box. 볼츠만의 엔트로피 방정식
\begin{eqbox} S = k \ln W ~~~~~ \text{(Boltzmann's entropy equation)}
\label{eq:boltzmann's_entropy_equation} \end{eqbox}

\begin{equation} \ln N! ~ \approx ~ N (\ln N) - N ~~~~~ \text{(Stirling's approximation)} \end{equation}