%%  chapter  %%%%%%%%%%%%%%%%%%%%%%%%%%%
%%%%  section  %%%%%%%%%%%%%%%%%%%%%%%%%
%%%%%%  subsection  %%%%%%%%%%%%%%%%%%%%
%%%%%%%%  subsubection  %%%%%%%%%%%%%%%%

%%  chapter  %%%%%%%%%%%%%%%%%%%%%%%%%%%
\chapter{Fluids}%

%%%%  section  %%%%%%%%%%%%%%%%%%%%%%%%%
\section{Fluids, Density, and Pressure}
\memo{chapter: 유체 \\ section: 유체, 밀도, 압력}%
\ 본문 내용

%%%%%%  subsection  %%%%%%%%%%%%%%%%%%%%
\subsection{What Is a Fluid?}
\memo{sub: 유체란 무엇인가?}%
\ 본문 내용

%%%%%%  subsection  %%%%%%%%%%%%%%%%%%%%
\subsection{Density and Pressure}
\memo{sub: 밀도와 압력}%
\ 본문 내용

%%%%%%%%  subsubection  %%%%%%%%%%%%%%%%
\subsubsection{Density}
\memo{subsub: 밀도}%
\ 본문 내용

%% eq box. 밀도
\begin{eqbox} \rho = \frac{\Delta m}{\Delta V}
\label{eq:density}\end{eqbox}

\begin{equation} \rho = \frac{m}{V} ~~~~~ \text{(uniform density)} \end{equation}

%%%%%%%%  subsubection  %%%%%%%%%%%%%%%%
\subsubsection{Pressure}
\memo{subsub: 압력}%
\ 본문 내용

%% eq box. 유체가 피스톤에 가하는 압력
\begin{eqbox} p = \frac{\Delta F}{\Delta A}
\label{eq:pressure_on_the_piston}\end{eqbox}

\begin{equation} p = \frac{F}{A} ~~~~~ \text{(pressure of uniform force on flat area)} \end{equation}

\begin{equation*} 1~\mathrm{atm} = 1.01 \times 10^5~\mathrm{Pa} = 760~\mathrm{torr} = 14.7~\mathrm{lb/in^2} \end{equation*}

%%%%  section  %%%%%%%%%%%%%%%%%%%%%%%%%
\section{Fluids at Rest}
\memo{section: 정지해 있는 유체}%
\ 본문 내용

%%%%%%  subsection  %%%%%%%%%%%%%%%%%%%%
\subsection{Fluids at Rest}
\memo{sub: 정지해 있는 유체}%
\ 본문 내용

\begin{equation} F_2 = F_1 + mg \end{equation}

\begin{equation} F_1 = p_1 A ~~~~~ \text{and} ~~~~~ F_2 = p_2 A. \end{equation}

\begin{equation*} p_2 A = p_1 A + \rho A g (y_1 - y_2) \end{equation*}

또는
%% eq box. 깊이 또는 높이에 따른 압력
\begin{eqbox} p_2 = p_1 + \rho g (y_1 - y_2)
\label{eq:pressure_as_a_function_of_depth_or_height}\end{eqbox}

\begin{equation*} y_1 = 0, \quad p_1 = p_0 ~~~~~ \text{and} ~~~~~ y_2 = -h, \quad p_2 = p \end{equation*}

%% eq box. 깊이에 따른 압력
\begin{eqbox} p = p_0 + \rho g h~~~~~\text{(pressure at depth } h\text{)}
\label{eq:pressure_at_depth_h}\end{eqbox}

\begin{equation*} y_1 = 0, \quad p_1 = p_0 ~~~~~ \text{and} ~~~~~ y_2 = d, \quad p_2 = p \end{equation*}

\begin{equation*} p = p_0 - \rho_{\text{air}} g d \end{equation*}

%%%%  section  %%%%%%%%%%%%%%%%%%%%%%%%%
\section{Measuring Pressure}
\memo{section: 압력의 측정}%
\ 본문 내용

%%%%%%  subsection  %%%%%%%%%%%%%%%%%%%%
\subsection{Measuring Pressure}
\memo{sub: 압력의 측정}%
\ 본문 내용

%%%%%%%%  subsubection  %%%%%%%%%%%%%%%%
\subsubsection{The Mercury Barometer}
\memo{subsub: 수은기압계}%
\ 본문 내용

\begin{equation*} y_1 = 0, \quad p_1 = p_0 ~~~~~ \text{and} ~~~~~ y_2 = h, \quad p_2 = 0 \end{equation*}

\begin{equation} p_0 = \rho g h \end{equation}

%%%%%%%%  subsubection  %%%%%%%%%%%%%%%%
\subsubsection{The Open-Tube Manometer}
\memo{subsub: 열린관 압력계}%
\ 본문 내용

\begin{equation*} y_1 = 0, \quad p_1 = p_0 ~~~~~ \text{and} ~~~~~ y_2 = -h, \quad p_2 = p \end{equation*}

계기 압력은
\begin{equation} p_g = p - p_0 = \rho g h
\label{eq:gauge_pressure}\end{equation}




%%%%  section  %%%%%%%%%%%%%%%%%%%%%%%%%
\section{Pascal's Principle}
\memo{section: 파스칼의 원리}%
\ 본문 내용

%%%%%%  subsection  %%%%%%%%%%%%%%%%%%%%
\subsection{Pascal's Principle}
\memo{sub: 파스칼의 원리}%
\ 본문 내용

%%%%%%%%  subsubection  %%%%%%%%%%%%%%%%
\subsubsection{Demonstrating Pascal's Principle}
\memo{subsub: 파스칼의 원리의 예}%
\ 본문 내용

%%%%%%%%  subsubection  %%%%%%%%%%%%%%%%
\subsubsection{Pascal's Principle and the Hydraulic Lever}
\memo{subsub: 파스칼의 원리와 유압지렛대}%
\ 본문 내용

%%%%  section  %%%%%%%%%%%%%%%%%%%%%%%%%
\section{Archimedes' Principle}
\memo{section: 아르키메데스의 원리}%
\ 본문 내용

%%%%%%  subsection  %%%%%%%%%%%%%%%%%%%%
\subsection{Archimedes' Principle}
\memo{sub: 아르키메데스의 원리}%
\ 본문 내용

%%%%%%%%  subsubection  %%%%%%%%%%%%%%%%
\subsubsection{Floating}
\memo{subsub: 떠 있는 물체}%
\ 본문 내용

%%%%%%%%  subsubection  %%%%%%%%%%%%%%%%
\subsubsection{Apparent Weight in a Fluid}
\memo{subsub: 유체에서의 겉보기 무게}%
\ 본문 내용

%%%%  section  %%%%%%%%%%%%%%%%%%%%%%%%%
\section{The Equation of Continuity}
\memo{section: 연속방정식}%
\ 본문 내용

%%%%%%  subsection  %%%%%%%%%%%%%%%%%%%%
\subsection{Ideal Fluids in Motion}
\memo{sub: 이상유체의 운동}%
\ 본문 내용

%%%%%%  subsection  %%%%%%%%%%%%%%%%%%%%
\subsection{The Equation of Continuity}
\memo{sub: 연속방정식}%
\ 본문 내용

%%%%  section  %%%%%%%%%%%%%%%%%%%%%%%%%
\section{Bernoulli's Equation}
\memo{section: 베르누이 방정식}%
\ 본문 내용

%%%%%%%%  subsubection  %%%%%%%%%%%%%%%%
\subsubsection{Proof of Bernoulli’s Equation}
\memo{subsub: 베르누이 방정식의 증명}%
\ 본문 내용