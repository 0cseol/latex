%%%%%%%%%%%%%%%%%%%%%%%%%%%%%%%%%%%%%%%%%
\documentclass[10pt]{book}
\input{../../structure/preamble.tex} % preamble(defs) 호출. 두 단계 올라가서 다시 내려옴
%\setupBib % 참고문헌 있을 시 주석 해제
\seolfmode = 4 % 모드 변경 - 1 출판용, 2 학생빈칸, 3 해답, 4 교사메모포함

%%%%%%%%%%%%%%%%%%%%%%%%%%%%%%%%%%%%%%%%% 문서 시작
\begin{document}
\linkbox % 언제나 목차로 이동하는 링크박스를 모든 페이지 하단에 삽입
\nopagenumber % 아직 페이지 넘버 표시하지 않기

\includeHead{title} % 타이틀
%\includeHead{preface} % 서문
\includeHead{toc} % 목차

%%%%%%%%%%%%%%%%%%%%%%%%%%%%%%%%%%%%%%%%% 본문
\newgeometry{top=25mm, bottom=25mm, inner=20mm, outer=60mm} % 학습지처럼 inner, outer로 여백 변경
\drawlinestrue % outer에 선 긋기 - inner, outer 있는 경우만 활용
\cleardoublepage % 양면인쇄를 위해 챕터를 포함한 plane 형식 문서가 오른쪽(홀수)로 시작되도록 함

\setcounter{page}{1} % 1페이지로 시작 - 변경 가능
\pagenumber % 지금부터 페이지 넘버 표시


%%%%%%%%%%%%%%%%%%%%%%%%%%%%%%%%%%%%%%%%% 챕터
%\loadchapter{00} % 설명문
\hiddenchapter{14} % 참조용 hidden
\setcounter{chapter}{17} % chapter를 17까지 카운트
\subimport{../chapters/18/}{chapter.tex}
%\subimport{../chapters/19/}{chapter.tex}
%\subimport{../chapters/20/}{chapter.tex}

\chaptercover{principles of physics}{seol} % chapter에서 마감하는 경우 커버

%%%%%%%%%%%%%%%%%%%%%%%%%%%%%%%%%%%%%%%%% 챕터 끝
\newpage
\drawlinesfalse % outer에 선 긋기 중단
\restoregeometry % 학습지 형식 용지 설정 끝. 기본용지로 변경


%%%%%%%%%%%%%%%%%%%%%%%%%%%%%%%%%%%%%%%%% 부록 있을 시 주석 해제
\appendix % 한 번만 선언(글 형식 바뀜)
%%% appendix 내부 파일에는 \appendix 명령 사용하지 않음

\chapter{실험 방법 요약}

본 연구에 사용된 실험은 고등학교 물리 교과서에 수록된 기본 실험 중 일부를 재구성한 형태로 진행됨. 실험의 목적은 학생들이 물리량 계산 시 유효숫자를 얼마나 일관되게 적용하는지를 관찰하는 데 있었음.

실험은 총 3단계로 구성되며, 각 단계는 실제 측정값을 바탕으로 물리량(속력, 밀도, 전류 등)을 계산하고 결과를 보고서 형식으로 작성하는 방식으로 이루어짐. 실험 조건은 최대한 학교 현장과 유사하게 구성하여 실제 수업 상황과의 연결을 고려함.

참가 학생들은 동일한 자료를 바탕으로 문제를 해결하되, 계산 결과의 표현 방식(자리수, 단위, 유효숫자 등)에서 차이를 보임. 이 차이를 기반으로 학생들의 개별적 인식과 실천을 분석함.

\chapter{자료 해석}
이런 식으로 사용할 수가 있다.
%%% 계열이 다르고, 부록의 양이 많은 경우 분리하여 작성할 수 있음

\chapter{예시자료}
하나 더 만들기
%\printBib % 참고문헌 있을 시 주석 해제


%%%%%%%%%%%%%%%%%%%%%%%%%%%%%%%%%%%%%%%%% 문서 끝
\end{document}