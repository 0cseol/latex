
\documentclass[doctor,korean,final]{cnu-ecs}

% kaist.cls 에서는 기본으로 dhucs, ifpdf, graphicx 패키지가 로드됩니다.
% 추가로 필요한 패키지가 있다면 주석을 풀고 적어넣으십시오,
%\usepackage{...}
%

%\usepackage{kotex}
\usepackage{multirow}
\usepackage{varwidth}

%\renewcommand\tablename{표}
%\renewcommand\figurename{그림}

\newtheorem{theorem}{Theorem}	

% @command title 논문 제목
% @options [default: (none)]
% - korean: 한글제목 | english: 영문제목
\title[korean]{가용성이 높고 규모 확장이 용이한\linebreak 망 관리 시스템 구조}
\title[english]{A Scalable and Highly Available\linebreak Network Management Architecture}

\author[korean] {이}{병 준}
\author[english]{Lee}{Byungjoon}

\advisor[major]{이 영 석}{Youngseok Lee}{signed}

\department{CSE}

% @command studentid 학번
\studentid{200560217}

% 논문제출일
\submitdate{2011}{4}{1}

% @command approvaldate 지도교수논문승인일
% @param   year,month,day 연,월,일 순으로 입력
\approvaldate{2011}{6}{15}

% @command refereedate 심사위원논문심사일
% @param   year,month,day 연,월,일 순으로 입력
\refereedate{2011}{6}{1}

% @command gradyear 졸업년도
\gradyear{2011}{8}

% 본문 시작
\begin{document}

% 목차 (Table of Contents) 생성
\tableofcontents

% 표목차 (List of Tables) 생성
\listoftables

% 그림목차 (List of Figures) 생성
\listoffigures

% 위의 세 종류의 목차는 한꺼번에 다음 명령으로 생성할 수도 있습니다.
%\makecontents


\chapter{서론}

\section{개요}

망(network)이 고도화됨에 따라, 효과적으로 망을 관리하는 일은 점점 
더 어려워지고 있다. 
특히 최근들어 두 가지 요인이 효율적인 망 관리를 방해하는 요인으로
대두되고 있는데, 첫번째는 망 규모(volume)이고, 두번째는 망의
복잡성(complexity)이다. 
하나의 망 안에 포설되는 장비의 대수가 증가함에 따라 망 관리에
드는 부담과 비용은 증가하며, 하나의 망을 구성하는 계층(layer)이나 
장비 종류가 다양해질 수록 망 관리는 어려워진다.

\chapter{관련 연구}

\section{다계층 구조}

흔히 다계층 구조(multi-tier architecture)라 부르는 계층구조는 
어떤 소프트웨어의 규모 확장성을 높이고자 할 때 가장 먼저 
고려되는 구조 가운데 하나이다. 
특히 3계층 구조(3-tier architecture)는 가장 보편적으로 쓰이고 있는
다계층 구조 중 하나로, 한 소프트웨어를 구성하는 데이터의 세 가지 쓰임새,
즉 정보제시(presentation 또는 view), 정보 모델링(model), 
정보 제어(controller)라는 세 가지 측면에 따라 소프트웨어의 구조를 설계함으로써
재사용성과 규모 확장성을 높이고자 하는 시도이다.

\chapter{결론}

오늘날의 망은 그 규모와 복잡성 면에서 날로 변화하고 있다.
클라우드 컴퓨팅 센터의 등장은 단순히 사용자가 자신의 데이터를 
인터넷 너머 어딘가에 저장할 수 있다는 것에 그치지 않고,
그 방대한 양의 데이터를 저장하기 위해 필요한 서버와 디스크들이 
어떻게 연결되어야 하는지, 그리고 그 연결이 만들어내는 새로운 형태의
대규모 망을 관리하려면 어떠한 망 관리 시스템이 필요한지에 
관한 새로운 물음들을 던지고 있다. 
기존의 망 관리 시스템들은 수만대에서 수십만대에 달하는
서버들과 망 장비들을 효과적으로 관리하기에 충분한가?
망이 규모의 경제를 실현하며 확대되는 것에 발맞추어,
망 관리 시스템 자체의 규모 확장성은 어떻게 확보해야 하겠는가?

% references section

% can use a bibliography generated by BibTeX as a .bbl file
% BibTeX documentation can be easily obtained at:
% http://www.ctan.org/tex-archive/biblio/bibtex/contrib/doc/
% The IEEEtran BibTeX style support page is at:
% http://www.michaelshell.org/tex/ieeetran/bibtex/
%\bibliographystyle{IEEEtran}
% argument is your BibTeX string definitions and bibliography database(s)
%\bibliography{IEEEabrv,../bib/paper}
%
% <OR> manually copy in the resultant .bbl file
% set second argument of \begin to the number of references
% (used to reserve space for the reference number labels box)
\begin{thebibliography}{1}

\bibitem{greenberg}
A. Greenberg, J. Hamilton, D.A. Maltz, and P. Patel,
\emph{The cost of a Cloud: Research Problems in Data Center Networks},
ACM SIGCOMM Computer Communication Review (CCR), 
Vol. 39, No. 1, pp. 68-73, January 2009.

\bibitem{wallin}
S. Wallin, V. Leijon, 
\emph{Telecom Network and Service Management: An Operator Survey}, 
MMNS, 2009.

\bibitem{hscalability}
Wikipedia, \emph{Horizontal Scalability},
\url{http://en.wikipedia.org/wiki/Scalability#Scale_horizontally_.28scale_out.29}

\end{thebibliography}

% 한글초록
\begin{summary}
현대의 망은 다양한 종류의 장비로 구성되며, 여러 계층을 포함하기 때문에 복잡도가 높고,
다량의 장비가 설치되는 대규모 망이다. 
\ldots
\end{summary}

% 영문초록 (abstract)
\begin{abstract}
Modern networks are large-scale, composed of many layers with tens of thousands of devices. 
Cloud computing data centers and multi-layered transport networks are 
examples of such networks.
\ldots
\end{abstract}

\chapter*{감사의 글}

감사의 글을 적으시면 되겠습니다.
감사합니다.

\begin{flushright}
\vspace{1cm}
이병준 배상
\end{flushright}

\end{document}


